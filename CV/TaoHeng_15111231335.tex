%should insert picture of QQ and wechat's QRcode
%求职必备!一份专业英文简历要注意哪些细节问题 ? https://mp.weixin.qq.com/s/rOdfkSoMQRQCNDSI5g6W1A


%# -*- coding:utf-8 -*-
%% start of file `template_en.tex'.
\documentclass[15pt]{moderncv}

\usepackage{fontspec,xunicode}
\setmainfont{Tahoma}
\usepackage[slantfont,boldfont]{xeCJK}
\usepackage{xcolor}                 % replace by the encoding you are using
\setCJKmainfont{Adobe Song Std L}
%\defaultfontfeatures{Mapping=tex-text}
%\XeTeXlinebreaklocale "zh"
%\XeTeXlinebreakskip = 0pt plus 1pt minus 0.1pt
% moderncv themes
\moderncvtheme[blue]{classic}                 % optional argument are 'blue' (default), 'orange', 'red', 'green', 'grey' and 'roman' (for roman fonts, instead of sans serif fonts)
%\moderncvtheme[green]{classic}                % idem
%\moderncvtheme[blue,roman]{hht}
% character encoding


% adjust the page margins
\usepackage[scale=0.85]{geometry}
%\setlength{\hintscolumnwidth}{3cm}						% if you want to change the width of the column with the dates
%\AtBeginDocument{\setlength{\maketitlenamewidth}{6cm}}  % only for the classic theme, if you want to change the width of your name placeholder (to leave more space for your address details
\AtBeginDocument{\recomputelengths}                     % required when changes are made to page layout lengths

% personal data
\firstname{陶恒}
\familyname{\huge{硕士研究生}}
\title{简历}               % optional, remove the line if not wanted
\address{湖南大学}{信息科学与工程学院}    % optional, remove the line if not wanted
%\mobile{123456789}                    % optional, remove the line if not wanted
\phone{086-15111231335}                      % optional, remove the line if not wanted
\email{franztaoheng@gmail.com}                     % optional, remove the line if not wanted
%\extrainfo{http://blog.csdn.net/hengfanz}
\extrainfo{https://github.com/hengfanz}
%\extrainfo{http://www.franztao.win/}
\photo[64pt]{th}                         % '64pt' is the height the picture must be resized to and 'picture' is the name of the picture file; optional, remove the line if not wanted
\quote{吾生也有涯而知也无涯}                 % optional, remove the line if not wante

%\nopagenumbers{}                             % uncomment to suppress automatic page numbering for CVs longer than one page


%----------------------------------------------------------------------------------
%            content
%----------------------------------------------------------------------------------
\begin{document}
\maketitle

\section{基本信息 Basic Information}
\cvline{性别}{男}
\cvline{籍贯}{湖南衡南}
\cvline{出生日期}{1992-07}
\cvline{现读学校}{湖南大学}	
\cvline{政治面貌}{中共党员}
%\cvline{期望工作地}{杭州,服从分配}
%\cvline{婚姻状况}{未婚}
%\cvline{身份证}{43042219920721901X}


\section{所获奖项Awards}
\cvline{硕士}{
2017年 湖南大学一等学业奖学金\newline
2017年 华为软件精英挑战赛 武长赛区三等奖64强\newline
2016年 第十三届全国研究生数学建模竞\textbf{二}等奖\newline
2016年 华为软件精英挑战赛 武长赛区二等奖32强\newline
2016年 湖南大学二等学业奖学金\newline}

\cvline{本科}{
2015年 湖南省优秀本科毕业生\newline
2014年 湖南省程序设计竞赛软件类三等奖\newline
2014年 湖南科技大学校级优秀班干部\newline
2014年 全国信息安全大赛三等奖\newline
2013年 ACM-ICPC亚洲区域赛长春赛区银奖\newline
2013年 湖南省ACM程序设计竞赛程序类二等奖\newline
2013年 湖南科技大学校级优秀学生\newline
2013-2014年 两次国家励志奖学金\newline}

\section{实习经历Internship}
\cvline{2017.7-9}{华为科技有限公司-网络工具开发部-软件工程师\newline \small{部署与维护工具割接精灵,对原有ping功能增强和优化。部署与维护工具配置转换,对华为原有qos,telnet,snmp功能一线新需求开发。完成两轮软件迭代开发周期,解决问题单14个,代码量1k+}}
\cvline{2016.7-8}{国防科技大学-系统软件技术培训学习}
\cvline{2015.7-9}{深圳乐美客技术有限公司公司-技术支持\newline \small{SBC硬件性能测试,wiki网站技术文档编辑,管理,整理和编辑论坛相关工作}}

\section{发表作品Publish}
\cvline{本科}{软著:可信水印相机(2015SR067514)}
\cvline{硕士}{论文:Divide And Conquer For Fast SRLG Disjoint Routing (DSN 2018)}
\section{项目经历Project Experience}
\cvline{项目起止日期}{\small 2012-05-01 ~ 2015-05-01}
\cvline{项目名称}{\small 数字水印在可信摄录设备中的应用研究}
\cvline{项目角色}{\small 算法设计员和优化员}
%项目地点
\cvline{主要职责与业绩}{\small 项目最终被评为湖南省优秀本科生研究项目,项目研发过程中,自己主要负责三方面的问题,核心算法瓶颈优化问题(二维离散余弦变换时间空间优化),算法运行平台代码的编写(基于安卓平台的运行代码)及软件图形界面的设计(合理设计UI使软件人性化).
水印算法的时间优化到原先的一半。安卓软件的UI图形界面无不稳定操作现象。}

\section{专业技能Profession Skills}
\cvline{数学与算法}{对\textbf{图论}算法有比较深的研究,对基本算法与数据结构熟悉,数学基础夯实,如矩阵论、数论。}
\cvline{程序语言}{掌握java(android),C++,C,assembly(X86),python,javascript(www.franztao.win),matlab,shell 等语言,具备良好语言的开发能力与技巧,能编写核心模块代码。}
\cvline{Linux}{熟悉linux基础知识,熟悉linux操作系统及linux环境编程,能在linux环境下完成日常运维工作}
\cvline{分布式网络}{熟悉TCP/IP协议,熟悉二层三层协议,熟悉VPN、MPLS以及路由交换协议(OSPF、BGP、ISIS等路由协议)原理,并有较强的分析故障排查能力和处理经验,具备基础HCNA能力,良好的移动通信知识。}
\cvline{软件定义网络}{熟悉SDN、OpenFlow协议、ODL、mininet等网络新技术,了解其基本理念和基本原理。}
\cvline{数据库}{掌握MySQL}
%\cvline{云计算}{熟悉OpenStack社区Keystone, Horizon, Ceilometer, Heat等Governance项目}
%\cvline{具体技术}{edit: vim,latex\newline
%datebase: mysql,sqlite\newline}
%pthread,MPI,OpenCV,Verilog,linggo,gurobi,GLPK,localsolver,make,shell,SPSS,wireshrak,tcpdump,mininet,opendaylight}
%Wireshark is a free and open source packet analyzer
%tcpdump is a common packet analyzer that runs under the command line.
%mininet,opendaylight
%Gephi is an open-source network analysis and visualization software package written in Java on the NetBeans platform
%wmware vSphere
%GNS
%TCL language
%openflow
%metasploit
%haskell
%docker
%MAVEN

\section{其他说明Other}
\cvline{专业描述}{\textbf{硕士阶段,计算机科学与技术}:研究生一年级主要研究基础查询算法(bloomfilter),现阶段研究生二年级主要研究软件定义网络\textbf{SDN多维流表算法}和\textbf{SRLG分离路径算法},网络虚拟化NFV中虚拟网络嵌入\textbf{VNE可生存性算法}。\newline
\textbf{本科阶段,信息安全专业}:主要从事与计算机安全相关的计算机技术,研究在图像中进行数字内容认证的各种脆弱性数字水印技术,包括:水印嵌入算法,水印盲检测算法,水印不可见性、安全性与鲁棒性的平衡研究。}
\cvline{学校实践}{本科一年级任职班级班长,本科二年级任职团支书,本科三年级至四年级任职班长,本科参加英语社团和程序设计集训队。
\newline 硕士二年级任职团支部书记。}
\cvline{自我评价}{\small 思维缜密,逻辑性强,具有挑战精神和前瞻性。\newline 良好的团队协作精神,具有良好的沟通能力,良好的书面和口头表达能力,较高的文档撰写水平(Latex)。\newline 喜欢编写代码,喜欢构思算法与代码,学习思考一些有趣的数学问题,喜爱拆卸维修电器,爱好运动,喜欢篮球运动与游泳运动。}

%{思维缜密,逻辑性强,沟通能力强,良好的团队协作精神,对网络技术有浓厚的兴趣,并有持续深钻技术的学习能力,}
%比较乐观外向,喜欢打羽毛球。
%
%栗子2正确打开方式
%我对自己的定位: 主攻前端,同时在其他方面打打辅助。我不希望过于依赖别人,即使没有后端没有设计没有产品经理,我依然想要把这个产品做到完美。毕竟全栈才能最高效地解决问题。
%我对工作的态度: 第一,要高效完成自己的本职工作。第二,要在完成的基础上寻找完美。第三,要在完美的基础上,与其他同事 互相交流学习,互相提升。工作是一种生活方式,不是一份养家糊口的差事。
%我怎样克服困难: 不用百度是第一原则,在遇到技术问题时我往往会去Google、Stack over flow上寻找答案。但通常很多问题 并不一定已经被人解决,所以熟练地阅读源码、在手册、规范甚至 REPL的环境自己做实验才是最终解决问题的办法。相信事实的结果,自己动手去做。
%怎样保持自己的视野:我一直认为软件开发中视野极其重要,除了在 Twitter 上关注业界大牛,Github Trending 也是每周必刷。 另外 Podcast、Hacker News、Reddit 以及TechRadar 也是重要的一手资料。保持开阔视野才能找到更酷的解决方案。
%我的优势: 热爱技术、自学能力强,有良好的自我认知。全面的技能树与开阔的视野,良好的心态、情商与沟通能力。
%我的劣势: 非科班出身没有科班同学对算法的熟练掌握,但我决定死磕技术,弥补不足。


\section{教育经历Education Background}
\subsection{硕士}
\cventry{2015--2018}{湖南大学}{计算机科学与技术-软件定义网络方向}{导师}{谢鲲}{成绩 top10}                % arguments 3 to 6 are optional
\subsection{学士}
\cventry{2011--2015}{湖南科技大学}{信息安全}{项目导师}{向德生}{成绩 top5}                % arguments 3 to 6 are optional

\section{语言情况Languages}
\cvlanguage{英语}{能进行熟练的英文读写,口语交流能力一般;}{CET-6 465}%Skill level
\cvlanguage{阿拉伯语}{一般}{自学}
\cvlanguage{粤语}{广州生活6年}{流畅}

%\section{Computer skills}
%\cvcomputer{category 1}{XXX, YYY, ZZZ}{category 4}{XXX, YYY, ZZZ}
%\cvcomputer{category 2}{XXX, YYY, ZZZ}{category 5}{XXX, YYY, ZZZ}
%\cvcomputer{category 3}{XXX, YYY, ZZZ}{category 6}{XXX, YYY, ZZZ}

\section{社会技能与资格证书Qualifications }
\cvline{社会技能}{\small C1机动车驾驶证(精通)}
\cvline{资格证书}{\small{中国计算机学会CCF软件能力认证(390 scores/full scores500)\newline
计算机技术与软件专业技术资格(水平)考试(软件设计师)\newline
计算机技术与软件专业技术资格(水平)考试(程序员)\newline}
湖南科技大学英语辅修结业证书}
%\cvline{hobby 3}{\small Description}

%\section{培训经历}
%\cvline{培训起止日期}{\small 2015-02-01 ~ 2015-03-01}
%\cvline{培训课程}{\small CCNA 思科认证网络工程师}
%\cvline{项目角色}{\small 算法分析员和编码员}
%培训机构 培训地点

\renewcommand{\listitemsymbol}{-} % change the symbol for lists

%\section{Extra 1}
%\cvlistitem{Item 1}
%\cvlistitem{Item 2}
%\cvlistitem{Item 3}
%\cvlistitem[+]{Item 3}            % optional other symbol

%\section{Extra 2}
%\cvlistdoubleitem[\Neutral]{Item 1}{Item 4}
%\cvlistdoubleitem[\Neutral]{Item 2}{Item 5}
%\cvlistdoubleitem[\Neutral]{Item 3}{}

%% Publications from a BibTeX file
%\nocite{*}
%\bibliographystyle{plain}
%\bibliography{publications}       % 'publications' is the name of a BibTeX file

%\begin{thebibliography}{99}
%%\bibitem{11} LaTeX入门与提高,高等教育出版社。
%湖南科技大学英语辅修结业证书\\
%\end{thebibliography}

\end{document}


%% end of file `template_en.tex'.
