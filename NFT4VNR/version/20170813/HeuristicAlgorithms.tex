\section{HeuristicAlgorithms}
In this section, we propose two heuristic algorithms for FD-SeVN problem, as well as an SeVN problem's algorithm with resources sharing consideration fitting for both FD-SeVN and FI-SeVN.

The general idea of the heuristic for FD-SeVN design is to consider the failure of primary nodes in Edit Grid sequentially, and in each step, compute the minimum additional resources needed to reassign the task graph based on an incremental approach (i.e., recovering from the current node failure should take
not only the survived primary nodes/links resources into consideration, but also the survived redundant resources reserved for previous node failure). After examining all the node failures, SeVN is constructed within the Edit Grid with the added redundant resources in each step.

However, as a matter of fact, computing the minimum additional resources needed to convert one attributed graph to another (hereafter called graph alignment problem, GAP) is an NP-complete problem which could be reduced from the Graph Edit Distance problem [26]. Therefore, we relax some constraints and propose two heuristic algorithms. In detail, we first decompose a graph to a multiset of star structures which retains certain structural information of the original graph. Then, the graph alignment cost could be approximated by the matching cost between two graphs based on their star representations.

This approach is elaborated as follows.

\subsection{FD-SeVN Algorithm}
\subsubsection{Graph Decomposition}
Graph Decomposition:Star Structure: A star structure s is an attributed, single-level, rooted tree which can be represented by a 4-tuple $s=(v,L^*,B^*,C^*,S^*)$, where r is the root node, $B^*$ is the bandwidth of each link associated with the root node, $C^*$ is the computation resource requirement of every nodes, $L^*$ is labels of nodes included in this star structure, $S^*$ is service type of every nodes . In this case, the node label is the task node mapped on it in the primary embedding. Edges exist between the root node and its adjacent nodes, and no edge exists among its adjacent nodes.

More exactly, for node $v_n$ in an attributed graph G (V, E, S, L),we can generate a star structure $s_n$ corresponding to $v_n$ in the following way: $star_n=(v_n,L^n,B^n,C^n,S^n)$ where $B^n=\{B^n_{n,u}|$for all $e_{n,u}\in E\}$, $C^n=\{C^n_{u}|$for all $e_{n,u}\in E\}$ and $L^n=\{l(v_u)|$for all nodes  adjacent with $n \}$. Accordingly, we can derive N star structures for a graph with N nodes. In this way, a graph can be transformed to a multiset of star structure.

Due to the particularity of star structure, the alignment cost between two star structures can be computed easily as below. For two star structures $star_x$ and $star_y$, the alignment cost of $star_x$ to $star_y$ is $\lambda(star_x,star_y)=\sum\limits_{l(v_u)\in L^x \cap L^y}[\beta|B^y_{y,u}-B^x_{x,u}|_0+\gamma |C^y_u-C^x_u|_0]+\sum\limits_{l(v_u)\in L^x - L^y}[\beta B^y_{y,u}+\alpha C^y_u]+\alpha|C_y-C_x|_0+\alpha * \delta(v_x,v_y)$. where $\delta$ and $|x|_0$ is defined as follows:

$\delta(v_x,v_y)\neq 0$ indicates that the label of root node $v_x$ should $v_x$ be reallocated as that for $v_y$.

In this way, both the task graph and the residual graph after specific virtual node failure could be represented as a multiset of star structures. With such a graph decomposition, an alignment matrix of these star structures could be constructed with the alignment cost definition above. Therefore, the proposed GAP
could be transformed to a bipartite graph matching problem which will be investigated in the following part.

\subsubsection{Bipartite Graph Matching}
Based on the discussion above, the computation of the graph alignment cost is equivalent to solving the vertices assignment problem, which is one
of the fundamental combinational optimization problems concerned with finding the minimum weight matching in a weighted bipartite graph. In our case, these two sets of vertices are the two multisets of star structures S1 and S2 , and the weight of the edge connecting stars in S1 and S2 is the alignment cost between the
corresponding two stars. Then, Hungarian algorithm could be applied to solve the bipartite graph matching problem in $O[|V |^3 ]$ time.




