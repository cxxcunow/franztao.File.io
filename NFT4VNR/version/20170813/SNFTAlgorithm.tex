
\section{1-SNFT graph construction for arbitrary graph}
The proposed algorithm of exploring node fault tolerant for arbitrary graph with specific flags is based on an extension of essential idea of Harary and Hayes' theorem\cite{harary1996node}. The heuristic algorithm applies the concept of Divide and Conquer to obtain a 1-SNFT graph with relatively(nonoptimal) less amount of edges. It works on arbitrary graph with specific flags consisting of elementary tree $T_e$ and circles $C_e$ which can be solved by our proposed optimal algorithms easily and concurrently such as Integer Linear Programmer(ILP) in sec.\ref{sec:SVN_ILP}. The elementary circles are the minimal circles which do not contain further circles inside.

Constructing the 1-SNFT graph for arbitrary flag label graph through proposed divided and conquer approach in Alg.\ref{alg:SNFTalg}. After initializing the $G^o$ as emptyset, the graph G is decomposed to individual elementary graphs $G_e$ in Alg.\ref{alg:DecomposeGraph}, whose 1-SNFT supergraphs $G^o_i$ are further constructed. After that the $G^o$ is multiply assigned by graph-theoreticly union between the $G^o$ and current $G^o_i$ iteratively.

The standard graph operations such as union, join, and composition\cite{west2001introduction} Binary operations create a new label graph from two initial ones $G_1 = (V_1, E_1)$ and $G_2 = (V_2, E_2)$, such as:
graph union: $G_1 \cup_g G_2 = (V_1 \cup V_2, E_1 \cup E_2)$.

The construct of $G^o_i$ with respect to $G_i$ is based on the approach from Sec.\ref{sec:SVN_ILP}.
%Here the intersection of graph theoretic is described using the example shown in Fig. pass.


\begin{algorithm}
\caption{Construct 1-SNFT(G,B) for arbitrary graph $G(V,E,S)$ and backup nodes $B(V,E)$}
\label{alg:SNFTalg}
\begin{algorithmic}[1]
\REQUIRE $G(V,E,S)$: arbitrary graph with specific flags; $B(V,S)$ backup nodes with specific flags.
\ENSURE 1-SNFT(G,B): specific node fault tolerant of graph $G$.
\STATE $G^o\leftarrow \emptyset$
\STATE $G_e\leftarrow$ DecomposeGraph($G(V,E,S)$)\ref{fig:DecomposeGraph_elementarygraph}
%\STATE AllMinimumCycle($G(V,E)$)
\FORALL{$G_i$ such that $G_i\in G_e$}
\STATE $G_i^o\leftarrow$1-SNFT-ILP($G$,$G_i$,B)
\STATE $G^o\leftarrow$($G_i^o \cup_g G^o$)
\ENDFOR
\FORALL{$i,j\in L;u,v\in V_e$}
\STATE $G^o\leftarrow G^o\cup E_{u,v}$
\ENDFOR
\RETURN $G^o$
\end{algorithmic}
\end{algorithm}





BCC-Decomposition: decompose a graph into two kinds of graph, tree and biconnected component graph.

ShortestHopPath: obtain the shortest hop path in Graph $subG_i$ of source node is v and destination node is u. Hence, there are many redundant cycles which is the same topological.
%\subsection{Decompose Graph Algorithm}
\begin{algorithm}
\caption{Decompose Graph}
\label{alg:DecomposeGraph}
\begin{algorithmic}[1]
\REQUIRE $G(V,E,S)$: arbitrary graph with specific flags.
\ENSURE $G_e$: elementary graphs of arbitrary graph $G$.
\STATE $G_e\leftarrow \emptyset$
\STATE $subG_i\leftarrow$ BCC-Decomposition($G(V,E,S)$)\ref{fig:DecomposeGraph_subbiconnectedgraph}
\IF{$subG_i$ is biconnected graph}
\STATE $C_e\leftarrow \emptyset$
\FORALL{$E_{uv} \in E $}
\STATE $P\leftarrow$ShortestHopPath($E_{uv}$)
\STATE $C_e\leftarrow C_e\cup(C=P\cup E_{uv})$
\ENDFOR
\STATE $C_e \leftarrow$ReduceRedundantCycle($C_e$)Alg.\ref{alg:ReduceRedundantCycle}
\STATE $G_e\leftarrow G_e\cup C_e$ %GraphMinimumCycles($subG_i$)
\ELSE
\STATE $G_e\leftarrow G_e\cup subG_i$
\ENDIF
\RETURN $G_e$
%\STATE request shortest hop paths of every adjacent nodes of the biconnected component
%\STATE add nodes with all node of  every paths then request minimum node cover set.
\end{algorithmic}
\end{algorithm}

%\begin{algorithm}
%\caption{Graph Minimum Cycles}
%\label{alg:GraphMinimumCycles}
%\begin{algorithmic}[1]
%\REQUIRE $subG_i$: Biconnected Component Graph
%\ENSURE $C_e$: Elementary Cycle $C_e$.
%\STATE $C_e\leftarrow \emptyset$
%\FORALL{$E_{uv} \in E $}
%\STATE $P\leftarrow$ShortestHopPath($E_{uv}$)
%\STATE $C_e\leftarrow C_e\cup_{s}(C=P\cup E_{uv})$
%\ENDFOR
%\STATE $C_e \leftarrow$ReduceRedundantCycle($C_e$)Alg.\ref{alg:ReduceRedundantCycle}
%\RETURN $C_e$
%\end{algorithmic}
%\end{algorithm}

ReduceRedundantIsormophismLabelGraph: Some cycles of cycle set $C_e$ is isomorphic in labeled graph $G$ so that we should sieve them for reducing unnecessary computation.

ConstructGraph: Remove the all edges of cycle, then add a new augmented node and link between all nodes of cycle and the augmented node as show in Fig.\ref{fig:DecomposeGraph_addnode4mimicycle}.

%MinimumBipartiteMatch: Every biconnected component of Graph $G^b$ is bipartite according to the construct method apparently. For obtaining the minimum number of cycles in order to contains all nodes of all cycles, ask minimum match of graph $G^b$, cycle of the node corresponding to match edge is result cycle.
MinimumSubsetCover: minimizing augmented nodes who link with all nodes of cycles is minimum subset cover problem, which is non-polynomial complete problem, and taking greedy algorithm solving the NPC problem\cite{cormen2009introduction}.
\begin{algorithm}
\caption{Reduce Redundant Cycle}
\label{alg:ReduceRedundantCycle}
\begin{algorithmic}[1]
\REQUIRE $C_e$: cycle set.
\ENSURE $C_{min}$: Irredundant Cycle Set.
\STATE $C^{b}\leftarrow \emptyset$
\STATE $C_{min}\leftarrow \emptyset$
\STATE $C_e\leftarrow$ReduceRedundantIsormophismLabelGraph($C_e$)
\FORALL{$G_i$ such that $G_i\in G_e$}
\STATE $G_i^b\leftarrow$ConstructGraph($G_i$)
\STATE $G^b\leftarrow$($G_i^b \cup_g G^b$)
\ENDFOR
\STATE $I\leftarrow$MinimumSubsetCover($G^b$)
\FORALL{$G_i$ such that $G_i\in G_e$}
\IF{i $\in I$}
\STATE $C_{min}\leftarrow$($C_{min} \cup G_i$)
\ENDIF
\ENDFOR
\end{algorithmic}
\end{algorithm}
\subsection{Consider different adjacent cycles}
$MCC$: $[mcc_{u,v}]_{(|V|+|B|)\times |V}$, incidence matrix of  . where
${mcc_{u,v}}=\left\{ \begin{array}{l}1\ if \\
0\ otherwise
\end{array} \right.$.
%\subsection{One Node Fault Tolerant With Service Algorithm}
%\begin{algorithmic}
%\FORALL{$G_i$ such that $G_i\in aG$}
%\STATE $M\leftarrow M\cup$ SUVIVABLE-GRAPH($G_i$)
%\ENDFOR
%\STATE for all node of subgraph, request optimal node-fault tolerant graph through ILP method.
%\end{algorithmic}