\begin{abstract}
Ensuring transmission survivability is a crucial problem for high-speed networks. Path protection is a fast and capacity-efficient approach for increasing the availability of end-to-end connections. The emerging SDN infrastructure makes it feasible to provide diversity routing in a practical network. For more robust path protection, it is desirable to provide an alternative path that does not share any risk resource with the active path. We consider finding the SRLG-Disjoint paths, where a Shared Risk Link Group (SRLG) is a group of network links that share a common physical resource whose failure will cause the failure of all links of the group. Since the traffic is carried on the active path most of time, it is useful that the weight of the shorter path of the disjoint path pair is minimized, and we call it  Min-Min SRLG-Disjoint routing problem.\revtao{ We prove this problem is NP-complete}. The key issue faced by SRLG-Disjoint routing is the trap problem, where the SRLG-disjoint backup path (BP) can not be found after an active path (AP) is decided. Based on the min-cut of the graph, we  design an  efficient algorithm that can  take advantage of existing search results to quickly look for the SRLG-Disjoint path pair. Our performance studies demonstrate that our algorithm can outperform other approaches with a higher routing performance while also at a much faster speed.
\end{abstract}

