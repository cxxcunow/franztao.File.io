\section{Algorithm}
\subsection{动态规划}
贪心和动态规划是有联系的,贪心是“最优子结构+局部最优”,动态规划是“最优独立重叠子结构+全局最优”。一句话理解动态规划,则是枚举所有状态,然后剪枝,寻找最优状态,同时将每一次求解子问题的结果保存在一张“表格”中,以后再遇到重叠的子问题,从表格中保存的状态中查找(俗称记忆化搜索)
\subsubsection{最大连续乘积子串}
给一个浮点数序列,取最大乘积连续子串的值,例如 -2.5,4,0,3,0.5,8,-1,则取出的最大乘积连续子串为3,0.5,8。也就是说,上述数组中,3 0.5 8这3个数的乘积30.58=12是最大的,而且是连续的。
\subsubsection{字符串编辑距离}
给定一个源串和目标串,能够对源串进行如下操作:
\begin{enumerate}
  \item 在给定位置上插入一个字符
  \item 替换任意字符
  \item 删除任意字符
\end{enumerate}
写一个程序,返回最小操作数,使得对源串进行这些操作后等于目标串,源串和目标串的长度都小于2000。
\subsubsection{格子取数问题}
有n*n个格子,每个格子里有正数或者0,从最左上角往最右下角走,只能向下和向右,一共走两次(即从左上角走到右下角走两趟),把所有经过的格子的数加起来,求最大值SUM,且两次如果经过同一个格子,则最后总和SUM中该格子的计数只加一次。\ref{fig:alg:gridGetNumber}
\begin{figure}
  \centering
  % Requires \usepackage{graphicx}
  \includegraphics[width=1.5in]{Algorithm/gridGetNumber}\\
  \caption{格子取数问题}\label{fig:alg:gridGetNumber}
\end{figure}
\subsubsection{交替字符串}
输入三个字符串s1、s2和s3,判断第三个字符串s3是否由前两个字符串s1和s2交错而成,即不改变s1和s2中各个字符原有的相对顺序,例如当s1 = “aabcc”,s2 = “dbbca”,s3 = “aadbbcbcac”时,则输出true,但如果s3=“accabdbbca”,则输出false。