% !Mode:: "TeX:UTF-8"

\addcontentsline{toc}{chapter}{附录A  发表论文和参加科研情况说明}
\chapter*{附录A~~~~发表论文和参加科研情况说明}
\setlength{\parindent}{0em}
\textbf{(一)发表的学术论文}
\begin{publist}
\item Kun Xie, Heng Tao, Xin Wang, Gaogang Xie, Jigang Wen, Jiannong Cao, Zheng Qin. Divide And Conquer For Fast SRLG Disjoint Routing[C]. DSN 2018: International Conference on Dependable Systems and Networks, Luxerbourg(CCF B).
%\item **. Divide And Conquer For Fast SRLG Disjoint Routing[C]. DSN 2018: International Conference on Dependable Systems and Networks, Luxerbourg(CCF B).
\item Kun Xie, Heng Tao, Xin Wang, Gaogang Xie, Jigang Wen, Jiannong Cao. Survivable embedded Virtual Network Design to Survive a Substrate Node Failure. (在投)
    %Communication Systems, Networks and Applications, Hongkong, 2010: 325-328. (EI DOI: 10.1109/ICCSNA.2010.5588732)
\end{publist}

\vspace*{1em}
\textbf{(二)申请及已获得的专利}
\begin{publist}
\item 陶恒,谢鲲. 一种求完全风险共享链路组分离路径对的方法及系统:中国。已具有国家知识产权局公开号,并已进入实审阶段。
%1234567.8[P]. 2017-04-25.
%\item **. 一种求完全风险共享链路组分离路径对的方法及系统:中国。%1234567.8[P]. 2017-04-25.
\end{publist}
%\vspace*{1em}
%\textbf{(三)参与的科研项目}
%\begin{publist}
%\item	XXX,XXX. XX~信息管理与信息系统, ~国家自然科学基金项目.课题编号:XXXX.
%\end{publist}
\vfill
\hangafter=1\hangindent=2em\noindent

\setlength{\parindent}{2em}
