% !Mode:: "TeX:UTF-8"

\chapter{总结和展望}
\section{本文工作总结与贡献}
本文针对网络故障的自身特点,对如何在SDN/NFV网络架构中提出时效快,低开销,高可生存性的保护算法,做出了多层次,多角度,多方位的深入研究和分析。本文针对SDN/NFV网络架构中可生存性算法研究做出了以下贡献。
\begin{enumerate}
  \item 首先,为了提高主路径传输的可生存性,全面研究了现在网络故障存在的各种故障类型特点,包括主动式和被动式。针对这些故障恢复机制方式,设计了快速的分而治之SRLG不相交路径对算法。
  \item 本文提出了一种拓扑图在存在陷阱问题的情况下求解Min-Min SRLG不相交路由问题的高效算法,为了降低搜索的复杂性,创新性提出了一种分而治之的解决方案,将原Min-Min SRLG 不相交路由问题划分为多个子问题,该子问题基于从AP路径上遇到陷阱问题时导出的SRLG冲突链路集。提出的算法利用现有的AP搜索结果SRLG冲突链路集,来并行执行实现更快的路径查找。
  \item 本文在一个多核CPU平台上使用合成的拓扑进行了真实的仿真模拟。仿真结果表明,在搜索速度比较下,该算法的查找性能优于其它现有算法。
  \item 分析虚拟网络嵌入问题的本质特征,从而提出适合星型分解动态规划节点嵌入的可生存性虚拟网络嵌入算法,以解决原有可生存性虚拟网络嵌入算法无法解决的时间复杂性和低资源利用率等问题。
  \item 本文在可生存性虚拟网络嵌入问题中,引入网络节点带有特定功能类型的限制条件,创新性提出一种星型分解动态规划节点嵌入的启发式算法,提出虚拟与物理星型图之间协调增广资源增加的权重设置机制,考虑网络资源的利用率和网络节点开启代价。
  \item   提出的算法能快速的实现虚拟网络嵌入的可生存性需求,仿真结果表明,提出的算法与其他现有算法效果比,虚拟嵌入可生存性请求的成功率更高,嵌入的物理资源消耗更低,物理资源的利用率更高。
\end{enumerate}



\section{未来研究工作与展望}
由于自身条件和时间的限制,对网络中可生存性算法的研究工作还不够深入也不够完善。在分析和总结本文工作的同时,未来还可以从如下几个方面继续研究:
\begin{enumerate}
  \item 对于SRLG不相交路径问题,可以考虑路径上的附属限制条件不单单只是一个累加性属性的权重值,可以同时考虑多限制条件,并且限制条件是多属性的\cite{linchuang2011},对不相交条件的限制不再是单单的共享风险链路组不相交,可以同时考虑点或者边不相交的混合情形。
  \item 本文提出的Min-Min SRLG不相交路径对算法只考虑的是单备份路径,算法可以通过迭代法和延伸SRLG冲突链路集的概念拓展到求多条Min-Min SRLG不相交路径的算法。
  \item SDN/NFV网络架构的集中式控制器模式,能易于实现过必经点的路由算法,则未来可以考虑路径对有必过节点或者链路的限制条件。
  \item 对可生存性虚拟网络嵌入问题研究中,未来可以考虑多个物理节点发生故障的情形,研究多个故障点发生的概率分布和影响特征,同样,我提出的算法能容易的实现多故障点的情形。未来可以考虑引入节点和链路更多的需求约束条件。
  \item 本文提出的星型分解动态规划节点嵌入的可生存性算法,没有考虑对虚拟链路映射的优化问题,可以在本文的算法的基础上做进一步改进,提出协调节点映射和链路映射的算法,来协调节点和链路映射两部分的资源增广和嵌入。
\end{enumerate}

%结论应是作者在学位论文研究过程中所取得的创新性成果的概要总结,不能与摘要混为一谈。
%学位论文结论应包括论文的主要结果、创新点、展望三部分,在结论中应概括论文的核心观点,
%明确、客观地指出本研究内容的创新性成果(含新见解、新观点、方法创新、技术创新、理论创新),
%并指出今后进一步在本研究方向进行研究工作的展望与设想。
%对所取得的创新性成果应注意从定性和定量两方面给出科学、准确的评价,分(1)、(2)、(3)…条列出,宜用“提出了”、“建立了”等词叙述。
