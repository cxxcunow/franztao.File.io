% !Mode:: "TeX:UTF-8"

\chapter{总结和展望}
\section{本文工作总结}
本文在SDN/NFV架构下提出了两种故障情形下的可生存性算法。

首先,本文提出了一种拓扑图在存在陷阱问题情况下求解Min-Min SRLG不相交路由问题的高效算法。为了降低搜索的复杂性,我们创新性提出了一种分而治之的解决方案,将原Min-Min SRLG 不相交路由问题划分为多个子问题,该子问题基于从AP路径上遇到陷阱问题时导出的SRLG冲突链路集。我提出的算法利用现有的AP搜索结果和并行执行来实现更快的路径查找。并且在一个多核CPU平台上使用合成的拓扑进行了广泛的模拟。仿真结果表明,在搜索速度比较下,该算法的查找性能优于其它现有算法。

其次,本文在可生存性虚拟网络嵌入问题中,引入网络节点带有特定功能类型的限制条件,创新性提出一种星型分割动态分配嵌入的启发式算法,在虚拟和物理星型图之间的权重设置上,考虑网络资源的利用率和网络节点开启的代价。我提出的算法能快速的实现虚拟网络嵌入的可生存性需求,仿真结果表明,我提出的算法与其他现有算法效果比,虚拟嵌入可生存性请求的成功率更高,嵌入的物理资源消耗更低,物理资源的利用率更高。

\section{未来研究工作与展望}
未来对Min-Min SRLG不相交路由问题的研究中,可以考虑对路径权重的考虑涉及多属性多类型的考虑,而且不相交的条件涉及点边一同不相交的情形。对可生存性虚拟网络嵌入问题研究中,未来可以考虑对物理节点发生故障的情形,和考虑节点和链路更多的需求约束条件。
%结论应是作者在学位论文研究过程中所取得的创新性成果的概要总结,不能与摘要混为一谈。
%学位论文结论应包括论文的主要结果、创新点、展望三部分,在结论中应概括论文的核心观点,
%明确、客观地指出本研究内容的创新性成果(含新见解、新观点、方法创新、技术创新、理论创新),
%并指出今后进一步在本研究方向进行研究工作的展望与设想。
%对所取得的创新性成果应注意从定性和定量两方面给出科学、准确的评价,分(1)、(2)、(3)…条列出,宜用“提出了”、“建立了”等词叙述。
