\chapter{分离路径问题}
网络在我们的日常生活中非常普遍。我们的身体由突触连接的神经元网络。我们的运输网络使我们能够轻松地往返于不同的地方。互联网是我们巨大的信息门户也是世界范围内的计算机网络。电网提供电力,而如果没有电力那我们现在社会可能会停止运作。我们的社交网络让我们和朋友和家人约会。由于网络的重要性,网络特性的多样性已经得到了广泛的研究,尤其是在图论领域。在图论中,网络被看作是一种通过链接互连的节点。节点表示网络的关节点,例如,通信网络中的路由器,海运网络,或交通网络中的城市。链接表示将关键点连接在一起的连接器,例如,通信网络中的电缆,海运中的贸易路线,运输网络中的网络或公路。图论中研究最多的课题之一是最短路径问题,即在网络中的两个节点,使得路径上链路权重之和最小化。传统的最短路径算法是Dijkstra算法[1]以及Bellman-Ford算法[2,3]。使用最短路径,信号可以在最小延迟之间交换在一个通信网络的两个路由器间,货物可以在海运网络中两个港口之间的以燃油成本最低的代价发送,而且在运输网络中我们可以更快地往返于城市之间。不相交路径(Disjoint Path)问题可以看作是一个扩展。最短路径问题。而不是只有一个单最短路径,几条不共享任何公共路径的路径计算链接(或节点)。提供不相交的途径网络流量将提高网络连接的可靠性,相应的网络生存性。网络生存性定义为网络的能力在网络组件存在的情况下提供持续的服务(例如,节点和/或链路)故障[4]。另一种变体不相交路径问题是不相交路径对问题,其中,而不是为单个对象找到多个不相交的路径。
对源节点和目标节点,计算单个路径。
对于每一对源节点和目标节点,这样的
道路是不相交的。
不相交路径有广泛的应用。例如,
在通信中具有多条不相交的通信路径
网络将提高其传输可靠性。通过
在多个不相交路径上并发发送通信量,则
路径的失败不会影响其他路径的性能。
路径,流量仍然会到达目的地。在运输中
网络,具有预先计算的不相交数。
路径将使卡车司机能够遵循不同的路径
改变风景而不是总是坚持最短的
路径。

%maritime network freights

本文的其余部分按以下方式组织。在不相交路径部分,给出了不相交路径的形式化定义。路径问题,讨论不相交的附加条件路径问题及其相应的复杂性解释几种有代表性的不相交路径算法。在基于可用性的不相交路径部分,我们将介绍路径可用性的概念及其与不相交路径的关系问题。最大不相交路径部分阐述在不相交路径可能部分重叠的情况下,而不是完全不相交。我们继续寻找不相交的地方域中多域网络上下文中的路径-不相交路径部分。因为多个链接(或多个节点)可能在类似的分担风险下同时失败,分担风险链接组(SRLG)-不相交路径部分介绍共享风险链接组的概念,并解释方法。确保不相交的路径不会同时失效由于单个链接(或节点)失败。风险也可能影响基于区域的网络,因此区域-不相交的路径本节讨论了几种基于区域的风险模型及其相应的风险模型。寻找区域的方法-不相交路径。阿与不相交路径问题对应的不相交路径对问题将在不相交路径对部分中讨论,的复杂性讨论了不相交路径对问题。最后,我们给最后一节对论文进行了简要的总结。
\section{完全分离路径}
\section{基于可靠性的分离路径}
\section{最大分离路径}
\section{范围分离路径}
\section{共享风险链路组分离路径}
\section{共享风险节点组分离路径}
\section{区域分离路径}
\section{分离路径对}
