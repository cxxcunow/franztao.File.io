% !Mode:: "TeX:UTF-8"
\section{相关的研究及发展趋势}
\subsection{共享风险链路组完全不相交路径对问题}
网络故障主要表现为链路故障,链路故障恢复策略可分为主动式和被动式两种\cite{kvalbein2006fast,qi2012research}。被动式策略在网络故障后自适应动态地进行全网资源重分配,但路由重新收敛花费较多的时间而不可接受。因此目前故障快速恢复研究以主动式策略为主,通过提前对网络进行资源规划和预留,使得故障时能迅速切换,如基于备份路径和基于多拓扑\cite{shand2010ip}的故障恢复技术。基于备份路径的故障恢复技术提供端到端路径重路由,在全局范围内进行流量分配,易于基于现有协议实现;多拓扑技术需要配置多个拓扑子层,路由存储消耗大。因此,备份路径技术是当前故障恢复领域研究的热点\cite{yang2014keep,wang2010r3,banner2010designing,suchara2011network,zheng2014cross}。

网络服务质量需求是多约束不同属性的条件,可以对这些条件加权值和归一化统一成单一约束条件,网络大部分工作流量集中主路径上,备份路径是当主路径出现故障时,临时使用来恢复故障的路径,所以对备份路径的服务质量需求并不需要特别苛刻,而应该尽量提高主路径的服务质量需求。而网络故障一般为链路故障,SDN/NFV架构下多租户数据中心Overlay 网络和传输层光网络的故已经不是单独一条链路的故障,而是拓扑图逻辑上几条链路同时故障,此时延伸到风险链路组SRLG的故障。

文献\cite{rostami2007cose,todimala2004imsh,xu2003trap}是获得完全不相交的SRLG路径对的启发式算法。一种链路不相交算法被扩展到SRLG不相交算法\cite{rostami2007cose},从而产生了冲突的SRLG冲突集算法。该算法将SRLG 不相交路径问题转化为Min-Min问题。文献\cite{gomes2010obtaining}提出了一种新的CoSE启发式算法,用于求解CoSE不相交的最小和问题。文献\cite{todimala2004imsh}的SRLG不相交路径问题也被描述为一个Min-Sum问题,并通过迭代启发式来求解,即迭代修正的Suurballe启发式(IMSH)。文献\cite{xu2003trap}提出了一种陷阱避免方案TA算法 ,其中考虑到的算法试图避免陷入陷阱(即算法无法找到SRLG不相交路径对的情况)。文献\cite{luo2007insights}提及的段保护用于避免陷阱。文献\cite{oki2002disjoint}考虑了一种基于k-最短路径的算法(加权SRLG 或WSRLG),其中根据与链路相关的SRLG 成员数将费用分配给链路。文中还考虑了根据链路与其他链路分担的风险分配链路成本的思想,其中提出了一种主动SRLG-多路径选择(ASPS)算法,该算法允许提高资源利用率(因为带宽资源在备用路径之间共享),并考虑到不同用户的区分可靠性要求。文献\cite{pan2006heuristics}中也考虑到共享资源保护的背景,文中提出了一种混合保护方案,该方案将共享路径保护和共享链路保护联合应用于具有风险共享链路组的WDM网络中。文献\cite{wang2007impairment}提出了两种不同的算法来计算给定的原点对和目标对的最优路径对,在这种情况下,路径对的最优性可以用两种不同的方式定义,从而导致两个不同版本的缺陷感知最佳路径对问题。文献\cite{cheng2007multiple}中考虑了一个不同的上下文,其中作者试图从一个预计算的路由表中找到一个AP和几个BP,并以最小的总附加带宽来承载特定的连接。文献\cite{rostami2007cose}提出了一种改进型CoSE的变体-modified-CoSE,在该算法中,如果不存在SRLG不相交路径对,则使用的SRLG路径对(其中AP最短且BP共享较少的SRLG)。文献\cite{silva2011heuristic}提出了TA算法\cite{xu2003trap}的一个变体指定的TA-Max。在TA-Max中迭代地考虑候选AP(为了增加成本),并为每个候选AP计算一个具有最小成本的BP,最后的解决方案是共享最少数目的公共SRLG的路径对。


现如今研究SRLG完全不相交路径问题的精确算法只有\cite{rostami2007cose,hu2003diverse,todimala2004imsh},而其他研究SRLG不相交路径问题的算法\cite{xu2003trap}都是启发式的非精确算法,即求解的路径对可能非最优路径对。算法\cite{todimala2004imsh}获得的路径对的优化目标是Min-Sum,即两条两路的权重和最小化,不保证主路径的权重最小化。在实际网络上,主路径承担主要的业务就需要其路径权重最小化。综上,求一对Min-Min SRLG完全不相交满足服务质量需求的路径对问题是现今网络故障可生存性算法的研究重点。

%故障恢复的本质在于维持故障链路承载流量的传输,因此应从流量持续传输角度解决故障恢复问题。大部分故障恢复工作集中在如何选择可靠备份路径上,且一般利用单一备份路径进行故障恢复。然而当流量超出备份路径可用带宽时,单一备份路径无法满足故障恢复的要求。受到这一启发,文献\cite{wang2010r3}将多路径技术引入到故障恢复中,采用多条备份路径共同承担流量,减少了流量的丢弃。在此基础上,文献\cite{banner2010designing}考虑了不同故障状况,通过将重路由流量分配给有跳数限制的多条备份路径,在网络投入运行之前就设计好应对各种故障场景的最低容量备份网络。文献\cite{suchara2011network}提出一种结合故障恢复与流量工程的网络结构,在多路径故障恢复基础上进行流量工程优化,其目的是进行负载均衡,且假设备份路径可用带宽满足故障恢复需求。文献\cite{zheng2014cross}提出一种跨层故障恢复模型,考虑了备份链路的可靠性,提升了故障恢复成功率。然而,上述故障恢复算法大都以最大化重路由为目标,未考虑恢复后的流量是否满足用户的需求。但是经由备份路径的重路由流量即使最终传输成功,由于链路过载、时延超时等原因,也无法满足业务的服务质量需求(Quality of Service,QoS),属于无效流量。虽然文献\cite{misra2009polynomial}提出了一种满足QoS 约束的自适应调整的多路径路由,但未考虑故障恢复问题。因此,目前已提出的大部分链路故障恢复算法不能很好地确保业务的服务质量。


\subsection{可生存性虚拟网络嵌入问题}
在考虑虚拟网络的嵌入过程中,常见的算法主要由两类:首先第一种是节点映射和链路映射彼此独立进行处理,将虚拟节点映射到物理节点上,然后由于此时已经映射的节点在底层拓扑的位置是己知的,因此接下来的链路映射就可以概述为多商品流问题\cite{even1975complexity}(Multi-commodity Flow Problem,MCF),这种算法比较简单,但是由于节点和链路 映射是独立进行的,只考虑到局部最优解无法获得全局最优解:第二种是节点和链路映射同时考虑,被称为虚拟网络嵌入协调过程,虽然它们都考虑到了全局情况,但是由于问题本身原理的限制,问题的复杂度规模是NP-hard,存在着成本比较高和结果不够精确的缺点。在虚拟网络的嵌入算法中,一般会考虑动态映射和静态映射两种情况\cite{fischer2013virtual},有效的嵌入算法在降低成本,提高资源使用率这一方面就显得尤为重要。



现有基于物理节点故障恢复的可生存性虚拟网络嵌入算法主要采用主动保护\cite{yu2011cost,wang2014survivable,sun2010efficient,hu2012location} 和被动恢复\cite{rahman2010survivable,qiang2014heuristic,bo2014dynamic}两类策略。主动保护策略在进行虚拟网络请求映射前为虚拟网络预置备份资源以供后续故障恢复使用,被动恢复策略在进行虚拟网络请求映射前不预置备份资源,在后续故障发生后再使用节点迁移或者现有的物理网络资源进行故障恢复。采用主动保护策略的算法包括FI-EVN\cite{yu2011cost}、FD-EVN\cite{wang2014survivable} LC-SVNE\cite{hu2012location}等,采用被动恢复策略的算法包括NB-SVNE\cite{bo2014dynamic}、MR-SVNE\cite{qiang2014heuristic}等。

可生存性虚拟网络嵌入算法的研究从基于主动保护和被动恢复两个方面提出了相应的故障恢复方法,但是依旧存在下面的一些不足:
\begin{itemize}
  \item 现有基于传统网络的SVNE算法中的主动保护策略比较固化,在物理网络无法为VNR提供全部所需备份资源的时候,会拒绝该VNR, 进而导致虚拟网络请求接受率和网络运营收益偏低,而SVNE算法中的被动保护策略易造成故障恢复效率低和故障恢复成功率抖动大的情况。
  \item 现有基于SDN网络的SVNE相关算法中采用的也是主动保护策略, 也存在主动保护策略的请求接受率和网络运营收益低的问题。虽然算法在此基础上增加了备份资源重分配机制\cite{yu2008rethinking}和虚拟网络映射资源来增网络运营收益加和网络资源利用率,但是资源动态调整机制不仅会增 加算法的时间复杂度,还易造成虚拟网络抖动,影响虚拟网络服务质 量。考虑到算法时间复杂度问题,本文暂不考虑使用资源重映射机制。
\end{itemize}

综上所述,目前的研究对于虚拟网络的保护策略不够灵活,或多或少都会出现性能短板问题,如主动保护策略中的虚拟网络请求接受率和收益低,被动恢复策略故障恢复效率低,传统网络下的SVNE算法无法对物理资源进行集中管控, 需要在物理设备间进行大量的通信,这会带来大量的通信开销,因此提出一个根据物理网络资源现状灵活实现虚拟网络冗佘备份的可生存性虚拟网络嵌入算法,可以来提高虚拟网络请求接受率、故障恢复效率和物理网络资源利用率,这非常有意义。

因此,为提高故障恢复成功率和虚拟网络请求接受率,使运营商的物理网络能够为租户提供更为可靠的虚拟网络服务,本文提出了星型分解动态规划节点映射的可生存性虚拟网络嵌入算法,此算法可以应用在主动保护和被动恢复两种策略的情形,与现有算法的比较,具有更高的可生存性接受率,资源利用率和收益,并且故障恢复效率更高,并且能拓展到应有于地域受限的虚拟网络嵌入问题和物理网络多节点故障的情形。




%为此,本文针对QoS 约束下的链路故障恢复问
%题进行研究,即在网络可用带宽和业务时延需求约
%束下进行最大化重路由流量问题求解。首先基于多
%备份路径策略建立概率关联故障模型和重路由流量
%丢弃优化目标,并构建QoS 约束的故障多备份路径
%恢复问题的数学优化模型。然后,设计QoS 约束的
%链路故障多备份路径恢复算法(Multiple backup
%Paths Recovery for link failure with QoS constrain
%algorithm, MPR-QoS) 对此问题进行求解。
%MPR-QoS 算法在构建单条备份路径时,利用改进
%的QoS 约束的k 最短路径法进行拼接,并以最大化p
%减少重路由流量丢弃为目标,且分配给高优先级链
%路更多的保护资源。此外还证明了算法的正确性并
%对时间空间复杂度进行了分析。最后,在NS2 仿真
%环境下从故障恢复率、重路由流量QoS 满足率、链
%路过载率和算法运行时间等方面验证了本文算法的
%优越性。
