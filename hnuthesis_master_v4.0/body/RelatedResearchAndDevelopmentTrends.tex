% !Mode:: "TeX:UTF-8"
\section{相关的研究及发展趋势}
\subsection{disjoint}
故障恢复的本质在于维持故障链路承载流量的
传输,因此应从流量持续传输角度解决故障恢复问
题。大部分故障恢复工作集中在如何选择可靠备份
路径上,且一般利用单一备份路径进行故障恢复。
然而当流量超出备份路径可用带宽时,单一备份路
径无法满足故障恢复的要求。受到这一启发,文献
[8]将多路径技术引入到故障恢复中,采用多条备份
路径共同承担流量,减少了流量的丢弃。在此基础
上,文献[9]提出一种结合故障恢复与流量工程的网
络结构,在多路径故障恢复基础上进行流量工程优
化,其目的是进行负载均衡,且假设备份路径可用
带宽满足故障恢复需求。文献[10]考虑了不同故障状
况,通过将重路由流量分配给有跳数限制的多条备
份路径,在网络投入运行之前就设计好应对各种故
障场景的最低容量备份网络。文献[11]提出一种跨层
故障恢复模型,考虑了备份链路的可靠性,提升了
故障恢复成功率。然而,上述故障恢复算法大都以
最大化重路由为目标,未考虑恢复后的流量是否满
足用户的需求。但是经由备份路径的重路由流量即
使最终传输成功,由于时延超时、链路过载等原因,
也无法满足业务的服务质量需求(Quality of Service,
QoS),属于无效流量。虽然文献[12]提出了一种满
足QoS 约束的自适应调整的多路径路由,但未考虑
故障恢复问题。因此,目前已提出的大部分链路故
障恢复算法不能很好地确保业务的服务质量。

\subsection{SeVN}
网络虚拟化的概念早己经兴起并且有了相当厚实的技术积累,现在出现/众 多的虚拟化平台,它们整合了众多技术摆脱了各自为政的局面,比较典型的有 Flowvisor、NVP、OpenContrail等。SDN是网络虚拟化极具吸引力的平台,因为每 个租户可以在其控制器上运行控制逻辑而不是在物理交换机上运行。在SDN中, 集中式控制器通过标准接口管理下层的交换机,用软件控制来自不同提供商的交 换机,另外SDN中控制转发面设备的协议OpenFlow提供了一个标准API,可以用于 设置数据包的转发规则,查询流量信息以及感知网络拓扑的变化。

VeRTIGO是一种新的网络虚拟化工具,可以允许开发人员在测试台实例化任 意的虚拟网络拓扑,提高了网络虚拟化的抽象层次,开发者可以设置自定义的虚 拟网络,并且手动实例化虚拟链路和虚拟节点或绘制所需的拓扑,这些拓扑可以 显示在OFELA(OpenFlow in Europe Linking Infrastructure and Applications)网络中, 它是基于SDN的架构框架,可以更好地处理链路故障。还有专家提出OpenRan架构, 它是-个通过虚拟化软件定义的无线接入网架构,这种架构实现了垂直方向的完 全虚拟和可编程性,可以提供开放和灵活的无线网络[61。

FlowN也是一种高效和可扩展的虚拟化解决方案,它基于SDN技术构建,以提 供对网络交换机的可编程控制。租户可以指定自己的拓扑、地址空间和控制逻辑。 FlowN架构利用数据库技术在虚拟网络和物理网之间进行映射。另外,FlowN使用 类似于基于容器虚拟化的共享控制器平台来高效地运行租户的控制器应用。这种设计决策为网络虚拟化带来了灵活、快速、可扩展的解决方案。使用FlowN,每个 租户可以运行自己的控制器应用程序。当然,并不是所有的租户都需要这么多控 制。需要简单网络表示的租户可以简单地选择默认控制器应用程序。
对于像云计算的共享基础设施,供应商应该完全虚拟化SDN以向租户表示网 络。为了支持种类各异的租户,云供应商应该让每个租户在自己的网络拓扑上指 定自定义控制逻辑,除了运行控制器应用程序之外,租户还指定网络拓扑。这使 得租户可以根据自己的需要来设计一个网络,例如为高性能工作负荷提供低延时 等。
微软也在2013年举办的开放网络峰会上介绍了将SDN技术引入云'汁算服务中 的解决方案。微软提出这种解决方案的原因是希望它的公有云服务WindowsAzure 可以作为企业的一个分支机构,而在该企业网络中的所有用户可以靠VPN访问公 有云服务上的各种应用以及业务m。该解决方案面临的挑战有两个:(1)每当用户有需求的时候,如创建一个新的租户或虚拟机,该服务要根据用户实际的需求,重新配置和调整网络资源;(2)公有云服务的规模要具有良好的可扩展性,以便 支持大数景级的网络节点和虚拟机。微软正是利用SDN虚拟化技术解决了这一难 题,该方案将数据转发平面和控制平面进行分离并进行了策略抽象,策略定义以 及下发都是通过软件的方式来完成的,而硬件只考虑可靠性和性能。Windows Azure 虚拟网络的核心是将用户自定义的地址空间映射到云服务供应商提供的虚拟机地 址,在这一过程中,用户的地址空间是由用户提出的网络配置需求经Azure Fronted 转化而来的,这一转换过程是由北向接口传给控制器的,两者之间的映射关系也 是由控制器建立的。
当前的LTE移动网络架构也面临着挑战和问题,在分析了大景移动网络中SON 和虚拟化的研宄工作后,有专家提出了移动网络中基于SDN和虚拟化的通用架构 SDVMN(SDN and virtualization in the mobile network),关注了每个载波网络采用 SDN和虚拟化时架构的变化。

在考虑虚拟网络的映射过程中,常见的算法主要由两类:第一种是节点映射 和链路映射彼此独立进行,将虚拟节点映射到物理节点上后,由于此时节点在底 层拓扑的位置是己知的,因此接下来的链路映射就可以概述为多商品流问题 (Multi-commodity Flow Problem,MCF),这种算法比较简单,但是由于节点和链路 映射是独立进行的,只考虑到局部,无法获得全局最优解:第二种是节点和链路 映射同时考虑,比较成功的算法有D-ViNe算法和vmxiFHb算法,虽然它们都考虑 到了全局,但是由于算法本身原理的原因,存在着结果不够精确,成本比较高的 缺点171。在虚拟网络的映射技术中,一般会考虑动态映射和静态映射两种情况,有 效的映射算法在提高资源使用率,降低成本这一方面就显得尤为重要,同样重要 的还有网络的负载均衡。
