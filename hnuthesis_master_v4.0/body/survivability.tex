\chapter{网络可生存性的基本原理}
为了增强网络性能、保证网络的健壮性,需要对网络中(本文以软件定义网络为研究背景)的许多问题进行优化设计,这些问题主要包括:网络可生存性设计问题;业务恢复问题等。随着网络业务爆炸性的增长及在各个领域的广泛应用和SDN网络的兴起,网络生存性问题已成为SDN网络关注的焦点。目前业内提出了许多关键词,比如可靠性、抗毁性等,它们的本质都是以网络生存性的研究为出发点。在本章中我们主要就网络可生存性问题展开讨论。

考虑到通信系统的今天的重要性和基础设施,网络应该是设计和操作的方式,使失败可以被破坏。例如,网络节点和/或链接可能会失败。恶意攻击,自然灾害,无意中的电缆削减,计划维修,设备故障,和等等。弹性,容错性,可生存性,可靠性,鲁棒性,和可靠的,是不同的术语已经被使用。网络社区捕捉通信能力面对时维护运行的制度网络故障。

\section{网络的可生存性概述}
前面提及到,在软件定义网络中控制层中心控制器管理每条信道,而当某条信道发生故障,则必须在短时间内恢复。因此,在软件定义网络网络控制层路由方面,对路由的生存性和可靠性变得尤为重要。

关于网络生存性(survivability)具体定义有多种说法\cite{al2009comparative}。1993年 Neumann\cite{hollway1993survivable}等人正式提出网络系统可生存性定义:在任意的不利条件下,基于计算机通信系统的应用应具有持续满足用户需求的能力,其中用户需求包含安全性、可靠性、实时响应和正确性等需求;至1999年 Ellison\cite{ellison1997survivable}等人进一步完善了网络系统生存性的定义:网络系统在遭受攻击、故障和意外事故的情况下及时完成任务的能力,属于网络完整性的一部分。

现实网络中网络故障是不可避免的,因此必须加强网络可生存性研究,可以采取对网络故障进行快速的检测、定位和恢复的方式,以保证网络的可生存性。

不幸的是,术语重叠的意思或包含歧义,如所指作者:Al-科威特等人。[1]。在本文中,我们将使用可生存网络一词是指网络,当组件失败,可以通过找到替代路径“生存”。这就避免了失败的组件。三种成分是需要达到生存能力。


\begin{enumerate}
  \item 网络连接,即网络应该是良好连通性(讨论了连通性)。
  \item 网络增强,即可能需要新的链路以增加网络的连接性。
  \item 路径保护,即寻找替代方案的过程。失败时的路径。
\end{enumerate}

\section{网络可生存性指标}
\begin{itemize}
  \item 业务请求拒绝率
  \item 平均网络负载
  \item 业务平均跳数
  \item 网络资源利用率
  \item 鲁棒性
  \item 故障恢复时间
\end{itemize}

\section{网络故障分类}
网络故障主要表现为链路故障,链路故障恢复
策略可分为主动式和被动式两种[张民贵, 刘斌. IP 网络的快速故障恢复[J]. 电子学报, 2008,
[4] 齐宁, 汪斌强, 王志明. 可重构服务承载网容错构建算法研究
[J]. 电子与信息学报, 2012, 34(2): 468-473. doi:
10.3724/SP.J.1146.2011. 00670]。被动式策略在
网络故障后动态自适应地进行全网资源重分配,但
路由重新收敛花费较多的时间而不可接受。因此目
前故障快速恢复研究以主动式策略为主,通过提前
对网络进行资源规划和预留,使得故障时能迅速切
换,如基于多拓扑[SHAND M and BRYANT S. IP fast reroute framework[P]
America]和基于备份路径的故障恢复技
术。多拓扑技术需要配置多个拓扑子层,路由存储
消耗大;基于备份路径的故障恢复技术提供端到端
路径重路由,在全局范围内进行流量分配,易于基
于现有协议实现。因此,备份路径技术是当前故障
恢复领域研究的热点[Towards k-link failure resilient routing,R3: resilient
routing reconfiguration,architecture for joint failure recovery and traffic
engineering,Designing low-capacity backup
networks for fast restoration[]。

在满足delay 约束的同时达到两条路径总的
花费(cost)最小.当给定的delay 约束针对两条路径的端到端延时总和时,问题被称为DCLDOP-I(delay
constrained link disjoint optimal paths),当给定的delay 约束针对路径对中每条路径的端到端延时时,问题被称为
DCLDOP-II.文献[Researches on the problem of link disjoint paths with QoS constraints]对DCLDOP-I 和DCLDOP-II 问题进行了建模,证明了这两种问题同属于NP 完全问题.文献
[4]针对DCLDOP-I 问题提出了两种近似求解算法.文献[Constrained shortest link-disjoint paths selection: A network programming based approach]研究了总延时受限下的k 条cost 最小链路分离路径
问题.文献[On the complexity of and algorithms for finding the shortest path with a disjoint]提出了Min-Min 问题,旨在求解两条满足QoS 约束的分离路径且满足较短的路径cost 最小.文献[Link-Disjoint shortest-delay path-pair computation algorithms for shared mesh restoration networks]
通过求解总延时最小的链路分离路径对来解决单链路失效后的路由恢复问题.
\section{网络故障恢复}
\section{网络保护策略}
\subsection{路径保护}
\subsection{链路保护}
\subsection{区段保护}
\subsection{其它保护方案}
\subsection{本章小结}

