\chapter{网络可生存性的基本原理}
为了增强网络性能、保证网络的健壮性,需要对网络中(本文以软件定义网络为研究背景)的许多问题进行优化设计,这些问题主要包括:网络可生存性设计问题;业务恢复问题等。随着网络业务爆炸性的增长及在各个领域的广泛应用和SDN网络的兴起,网络生存性问题已成为SDN网络关注的焦点。目前业内提出了许多关键词,比如可靠性、抗毁性等,它们的本质都是以网络生存性的研究为出发点。在本章中我们主要就网络可生存性问题展开讨论。

考虑到通信系统的今天的重要性和基础设施,网络应该是设计和操作的方式,使失败可以被破坏。例如,网络节点和/或链接可能会失败。恶意攻击,自然灾害,无意中的电缆削减,计划维修,设备故障,和等等。弹性,容错性,可生存性,可靠性,鲁棒性,和可靠的,是不同的术语已经被使用。网络社区捕捉通信能力面对时维护运行的制度网络故障。

\section{网络的可生存性概述}
前面提及到,在软件定义网络中控制层中心控制器管理每条信道,而当某条信道发生故障,则必须在短时间内恢复。因此,在软件定义网络网络控制层路由方面,对路由的生存性和可靠性变得尤为重要。

关于网络生存性(survivability)具体定义有多种说法\cite{al2009comparative}。1993年 Neumann\cite{hollway1993survivable}等人正式提出网络系统可生存性定义:在任意的不利条件下,基于计算机通信系统的应用应具有持续满足用户需求的能力,其中用户需求包含安全性、可靠性、实时响应和正确性等需求;至1999年 Ellison\cite{ellison1997survivable}等人进一步完善了网络系统生存性的定义:网络系统在遭受攻击、故障和意外事故的情况下及时完成任务的能力,属于网络完整性的一部分。

现实网络中网络故障是不可避免的,因此必须加强网络可生存性研究,可以采取对网络故障进行快速的检测、定位和恢复的方式,以保证网络的可生存性。

不幸的是,术语重叠的意思或包含歧义,如所指作者:Al-科威特等人。[1]。在本文中,我们将使用可生存网络一词是指网络,当组件失败,可以通过找到替代路径“生存”。这就避免了失败的组件。三种成分是需要达到生存能力。


\begin{enumerate}
  \item 网络连接,即网络应该是良好连通性(讨论了连通性)。
  \item 网络增强,即可能需要新的链路以增加网络的连接性。
  \item 路径保护,即寻找替代方案的过程。失败时的路径。
\end{enumerate}

\section{网络可生存性指标}
\begin{itemize}
  \item 业务请求拒绝率
  \item 平均网络负载
  \item 业务平均跳数
  \item 网络资源利用率
  \item 鲁棒性
  \item 故障恢复时间
\end{itemize}

\section{网络故障分类}

\section{网络故障恢复}
\section{网络保护策略}
\subsection{路径保护}
\subsection{链路保护}
\subsection{区段保护}
\subsection{其它保护方案}
\subsection{本章小结}

