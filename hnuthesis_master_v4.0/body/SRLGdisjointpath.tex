\chapter{共享风险链路组分离路径算法研究}
\section{算法概述}
共享风险链接组(SRLG)是一组链路共享相同的一个组件,该组件的故障会导致在这个组里所有链接的发生故障。就路径保护而言,尽管某些链路或者节点分离路径算法\cite{suurballe1984quick,bhandari1997optimal,li1990complexity,guo2003link,xu2004finding,beshir2011variants,guo2013finding,hu2003diverse}已经提出来,SRLG分离路径 问题是比较棘手的,而且这些原有研究是限制在一定领域的。比如当每个SRLG只包含一条链路时,这个SRLG分离路由问题可以简化为链路分离路径问题,而通过节点分离方法(node split method)\ cite{ford2015flows}节点分离路径问题可以转化为链路分离路径问题。因为SRLG组通常包括的链路超过一条并且网络中的链路通常可以属于多个SRLG组里,以至于求一对SRLG 分离路径问题比求一对链路或者节点分离路径问题要困难得多。

为了解决SRLG分离路径问题,一种可能的方法是0-1整数线性规划(ILP)\cite{hu2003diverse},通过分支限界法(branch-and-bound)来搜索来选择最优的主路径和备份路径。该方法时间复杂度高,不适用于大型网络。为了降低算法的复杂度,基于APF的启发式算法\cite{oki2002disjoint,li2002fiber,eppstein1998finding}能够求Min-Min SRLG分离路径问题的近似最优解。首先使用Dijkstra算法(或任何其他最短路径算法)求出主路径,求主路径时不考虑其相应的备用路径情况,在删除AP沿线的链路并且与AP共风险的节点和链路后,再利用最短路算法求的备用路径。

然而,使用APF启发式算法的有一个主要缺陷,一旦求得路径AP后也可能无法找到相对的SRLG分离路径BP,即使网络中确实存在一对分离路径。这就是所谓的“陷阱”问题,即使稠密网络中\cite{laborczi2001solving}这也是可能发生,在一个稀疏连接的网络中当然不能被忽略。陷阱有两种:不可避免的陷阱和可避免的陷阱。不可避免的陷阱是受拓扑约束的,任何算法都无法解决。如果网络不是2-边连通度的,则没有算法可以保证在拓扑中存在两个SRLG分离路径。另一方面,当两个节点之间存在SRLG分离路径对,但由于路由算法的缺陷而找不到时,就会出现一个可避免的陷阱。在本章中只考虑了可避免的陷阱。

对简单的APF算法的扩展,提出了KSP(K-最短路径)算法来处理节点/链路分离路径的陷阱问题。虽然它是处理陷阱问题最有效的算法之一,但它在大型网络中的性能受到影响,因为KSP 可能会涉及多路径搜索测试(K测试),直到它找到分离路径。当前候选的路径AP遇到陷阱问题后,仅根据路径长度选择下一个要测试的候选AP,而不考虑当前候选AP的那条链路(或那些链路)导致查找分离路径BP失败。因此,为了找到一对分离路径对,需要对大量的路径进行测试,这就引入了KSP算法中与K相关的时间复杂度。对于遇到陷阱问题的AP,我们应用从AP 路径导出的SRLG冲突链路集来指导将来的AP路径测试。这在很大程度上有助于减少寻找替代路径的时间复杂度。

其他SRLG-不相交路由算法,包括[27],[28],[29],[30],[31],搜索最大的SRLG不相交的路由路径,共享的公共链路的最小数目。由于AP和BP可能具有相同的风险,通过这种方法找到的解决方案是不可靠的。我们的算法目标是寻找完整的SRLG不相交路径。[32]中的工作试图找到完整的SRLG不相交路径。它减少了问题的搜索空间,加快了路径搜索的速度。然而,它可能会以较大的代价返回路径,因为在削减后的空间可能会失去最优解。相反,为了大大加快搜索过程,我们利用发现的冲突链接集将原问题划分为多个子问题,这些子问题可以并行执行。因此,我们的算法可以运行得更快,返回路径的成本非常低。

\section{问题描述}
\section{整数规划形式化}
\section{时间复杂度}

\section{trap问题}

\section{分而治之的快速共享风险链路组分离路径算法}
\subsection{分而治之}
\subsection{SRLG冲突链路集合}
\subsection{算法步骤}
\subsection{实验环境与评价指标}
\subsection{算法性能评估及比较}
