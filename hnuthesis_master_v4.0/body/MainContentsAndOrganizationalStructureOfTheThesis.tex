% !Mode:: "TeX:UTF-8"
\section{论文的主要研究内容和组织结构}
\subsection{论文的主要研究内容}
SDN作为新型网络架构,网络虚拟化作为未来网络研宂的重要领域,两者的 结合更具研究意义。在网络虚拟化的过程中,不可避开的一个技术就是映射算法 的研究,映射算法主要考虑的是虚拟网络节点和链路向底层物理网的节点和链路 的映射,考虑如何为虚拟网络在底层物理网中找到满足条件且最优的路由,并且 尽可能地提髙网络资源的利用率。鉴于此,本文做了如下研究:

1)首先,研究SDN虚拟网络技术,对SDN网络虚拟化的环境进行了研究,并 且探讨/虚拟网络向底层物理M进行映射时的不同映射算法。

2)	在虚拟网络向底层物理网络映射时,考虑节点和链路的映射以及路由算法 的设计,在路由算法的设计中,假设节点己经映射,路由算法主要是为两个物理 节点寻找满足条件的路径,由此设定了两个大的场景:无约束条件路由寻找和有 约束条件下路由的寻找。在无约束路由寻找这一情况下,分别设计算法实现了代 价和最小的最短分离路径对的获得以及尽童分离的最短K条路径。在有必经约束这 一条件下,也分别设计算法实现了寻找最短路径以及代价和最小的最优路径对。

3)	在虚拟网络映射过程中,为了得到最优路由,我们主要考虑的是跳数和花 费问题,但是在实际网络环境中,只是得到了节点之间的最优路由,这种最优是 局部的,导致整个网络的资源利用率也不一定高,可能还会出现网络拥塞的情况。 所以,本文设计了基于虚拟网络节点和链路的不同映射方案来保证网络负载均衡 并且高效利用网络资源。

现有传统数据中心网络下的生存性虚拟网络映射方案,由于网络结构问题无 法实现网络资源集中化管控,所以本文将考虑结合SDN网络的集中化管控特性, 基于SDN网络进行算法研宄。现有基于SDN网络的SVNE算法中的主动保护 策略对于虚拟网络请求接受率和网络运营收益性能方面存在改进之处,而被动保 护策略对于算法故障恢复效率和故障恢复成功率等性能方面存在可改进之处,为 改善算法的虚拟网络请求接受率、网络运营收益和故障恢复效率等性能指标,本 文旨在提出一种SDN网络环境下基于资源感知备份策略的SVNE算法,算法将 结合原有的主动保护策略和被动恢复策略的优点,以改进上述算法指标。
本课题将结合位置约束的概念,以虚拟N络请求的接受率、物理网络资源利 用率和故障恢复率等指标作为算法性能评价B标,提出一个基于资源感知备份策略的生存性虚拟网络映射算法。为了实现课题目标,需要进行以下研宄工作:

1.	调研现有的生存性虚拟网络映射算法,以及基于SDN的虚拟网络映射算 法和生存性虚拟网络映射算法,分析不同算法的优缺点,以及存在的共 性问题。

2.	提出一种SDN网络环境下基于资源感知备份策略的生存性虚拟网络映 射算法,算法在为虚拟网络的待备份虚拟节点提供备份资源的时候,通 过检测当前物理网络的剩余资源情况,可仅为满足备份资源需求的虚拟 节点提供备份资源,以改进现有主动保护策略存在的虚拟网络请求接受 率低和被动恢复策略存在的故障恢复效率较低等性能问题。

3.	设计和搭建仿真平台,对提出的映射算法进行仿真,并实现相关的对比 方案。
4.	对实验结果进行分析,评估本文方案和对比方案的性能优劣。


\subsection{论文的组织结构与贡献}
本文依据SDN、路由算法以及分离路径。。等方面的研究现状与发展趋势,并结合自身对以上的研究和探索,将本文分为五章,各章主要内容如下:
本文依据SDN、网络虚拟化技术、路由算法以及负载均衡等方面的研究现状 与发展趋势,并结合自身对以上内容的研究和探索,将本文分为五章,各章主要 内容如下:
第一章为绪论部分,主要介绍了本课题的研宄背景及其意义并且简要地分析 了国内外研宂现状和发展趋势。
第二章介绍了网络虚拟化的相关技术,概述了SDN网络虚拟化环境和SDN网 络虚拟化技术特征,接着洋细介绍了虚拟网络映射算法的相关研宄。
第三章主要是SDN虚拟网络映射中路由算法的设计与实现,设计了两套路由 算法:其一是寻找链路分离的路径,设计算法找到代价和最小的完全分离路径对, 同时也设计算法实现了尽量分离的最短K条路径:其二是在有必经节点约束的情况 下找到最短路径,同时也设计算法在有必经节点约束的限制下找到代价和最小的 最短分离路径对,这些算法的设计都是在节点己经映射后,为节点间寻找满足要求的路径。
第四章设计了基于虚拟网络节点和链路的不同映射方案来保证网络的负载均 衡。
第五章对现有的工作和研究成果进行了总结,并进一步对其中的某些技术问 题探讨了改进方案。
\begin{itemize}
  \item 第一章为绪论部分,主要介绍了本课题的研究背景及其意义并且简要地分析了国内外研究现状和发展趋势。
  \item 介绍
  \item 介绍
\end{itemize}
