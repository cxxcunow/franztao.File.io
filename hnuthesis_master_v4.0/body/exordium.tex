% !Mode:: "TeX:UTF-8"

\chapter{绪论}
% !Mode:: "TeX:UTF-8"

\chapter{网络可生存性的基本原理}
为了增强网络性能、保证网络的健壮性,需要对网络中(本文以软件定义网络为研究背景)的许多问题进行优化设计,这些问题主要包括:网络可生存性设计问题;业务恢复问题等。随着网络业务爆炸性的增长及在各个领域的广泛应用和SDN网络的兴起,网络生存性问题已成为SDN网络关注的焦点。目前业内提出了许多关键词,比如可靠性、抗毁性、鲁棒性和可信赖性等,它们的本质都是以网络生存性的研究为出发点。在本章中我们主要就网络可生存性问题展开讨论。



\section{网络的可生存性概述}
软件定义网络中控制层中心控制器管理每条信道,而当某条信道发生故障,则必须在短时间内恢复。因此在软件定义网络网络控制层路由方面,研究路由的生存性变得尤为重要。

关于网络可生存性(survivability)具体定义有多种说法\cite{al2009comparative}。1993年 Neumann\cite{hollway1993survivable}等人正式提出网络系统可生存性定义:在任意的不利条件下,基于计算机通信系统的应用应具有持续满足用户需求的能力,其中用户需求包含安全性、可靠性、实时响应和正确性等需求;1999 年 Ellison\cite{ellison1997survivable}等人进一步完善了网络系统可生存性的定义:网络系统在遭受攻击、故障和意外事故的情况下及时完成任务的能力,属于网络完整性的一部分。


可生存性是与安全性相关的一个重要属性, 它反映了网络系统面临入侵和攻击等安全破坏威胁时仍能恢复正常服务的能力\cite{jha2000survivability}。 关于可生存性的研究最早开始于军事领域\cite{frank1974survivability}。 随着安全破坏发生频度的逐渐增多, 可生存性越来越多地引起人们兴趣和关注。 网络系统的可生存性的量化分析和验证是一个重要的基础工作。


可生存性反映了网络的整体性能, 是网络研究
的一个基本目标。 现今的网络环境下, 造成服务失效
的原因很多, 例如, 系统运行过程中出现的软硬件故
障、恶意攻击或破坏或者自然因素。 可生存性分析需
要分析失效的现象, 剖析影响可生存性的本质特征,
建立可量化的评价模型。 针对不同的网路环境和不同
的评价原则人们给出了各种可生存性的定义[ 51 ~ 53] 。
人们使用各种方法来分析网络的可生存性, 其
中大部分是基于图论和随机模型。 网络可生存性的
分析要建立在攻击模型的基础上。 借助本文中上面
章节中提到的各种攻击模型可以进一步分析可生存
性。 目前, 这部分的研究工作仍然是针对局部网络展
开, 还没有形成系统的评价方法和实现机制。 文献
[ 50] 对网络系统建立网络流图模型(netw ork f low
g raph), 并利用max-flow , min-cut 算法得到了确定
可生存性的评价指标。 文献[ 25] 利用模型检测技术
生成系统的情景图(scenario graph), 在节点上赋予


考虑到今天通信系统和基础设施的重要性,网络应该在设计和操作上考虑到系统或者设施出现故障能够被解决。例如,由于恶意攻击,自然灾害,突然间的电缆断裂,计划维修,设备故障等等,网络节点/链接可能出现故障。弹性,容错性,可生存性,可靠性,鲁棒性和可信赖性,是已经被使用的不同术语。这些术语是面对网络故障时网络维护运行保持通信能力的术语。如作者\cite{al2009comparative}所指,不同的术语有重叠的意思和一定的歧义。在本文中,我们将使用可生存网络一词是指在网络当一个网络组件出现故障,可以通过找到替代路径来避免使用故障的网络组件。

随着人们对生存性技术的关注,国内外的许多研究机构都对其投了大量的人力、物 力,其研究重点可概括为可生存性的基本概念\cite{kuipers2012overview}、可生存性体系结构、系统模型\cite{luxinhua2006}、 系统分析与设计、系统工程方法和工具、生存性风险评估\cite{linxuegang2006} 等。


就国内目前的研究来看,网络生存性设计主要基于两种模式\cite{hanjianjun2007},即基于入侵使用情景 的重新设计和基于入侵容忍技术的生存系统设计方法。简单来说,第一种模式是在故障产 生后对系统模型的重新开发,目前还处于研究阶段;而第二种模式采用了容忍入侵的策略,它具有高效的监测系统,一旦发现故障,便调用冗余资源进行系统异常修补。第二种模式 的代价较第一种模式要高一些,并且现阶段的研究还不是很成熟,目前比较经典的是基于 入侵检测技术的设计模式,它通过对网络故障的快速检测、定位和恢复使其可靠性提高。

现阶段,网络生存性研究已经得到了很大重视,很多知名学者就该问题的不同方面作了深入研究,问题涉及故障分类、生存性建模分析、故障恢复技术、生存性网络规划 等许多子领域,并取得了一些成果。三种方法是能使网络达到可生存性能力。
\begin{enumerate}
\item 网络连通,即可生存性好的网络应该是高连通性的。
\item 网络增强,即可能需要新的链路以增加网络的连通性。
\item 路径保护,即寻找替代方案的过程,失败故障时的替代路径。
\end{enumerate}


现实网络中网络故障是不可避免的,因此必须加强网络可生存性研究,可以采取对网络故障进行快速的检测、定位和恢复的方式,以保证网络的可生存性。
%An Overview of Algorithms for Network Survivability

\section{网络可生存性指标}
本文在进行网络生存性设计时,要达到的最理想的状态是:对于给定的网络拓扑结构, 能够在最短的时间内使故障元素获得最大程度的恢复,并同时保证最大的资源利用率。然而由于事件的互斥性,很难同时实现所有这些要求,所以需要根据不同的业务或用户需求 以及网络本身的特点,采取相应地措施,从而满足网络的生存性指标要求。

常用的网络生存性指标主要有:
\begin{itemize}
  \item 故障恢复时间:是指从网络故障发生时,到业务恢复正常传输所需的时间。该 项指标最直接,也最能体现网络的生存性能力。众所周知,对于网络用户来说,传输过程 是透明的,他们所能感受到的就是网络提供的业务服务质量,而其中最敏感的因素便是故 障恢复时间。
  \item 业务请求拒绝率:是指被拒绝的网络业务数最与总体业务请求数量的比率。可生存性网络研究的目标是最小化请求拒绝率。
  \item 平均网络负载:网络负载是指网络链路中的流量负荷,一且链路负荷过重,将直 接影响整体网络的迕通性。因此,最小化网络负载将有利于网络生存性的研究。
  \item 业务平均跳数:某些网络有时以步长来衡量代价耗费,因此最小化路径长度是网 络优化指标之一。
  \item 网络资源利用率:这是衡量任何一个网络优劣的可靠指标,提高网络利用率是网络运营商追求利益的根本。
  \item 鲁棒性:即健壮性或者可生存性,是指经历过一次网络故障后,网络再次承受故障的能力。它主要用来衡量网络业务的可持续性。
\end{itemize}
其中,网络资源利用率、业务请求拒绝率和平均网络负载是几个最常用的评价指标。


\section{网络故障分类}
由于遭受破坏的因素及网络元素不同,网络故障也是多种多样,按照光网络中故障的 表现方式不同可分为软故障和硬故障;按照故障产生的元素不同可分为信道故障、节点故障和链路故障。概念解释如下:

所谓硬故障是指某些意想不到的突发事件致使传输信道中断的事件,如地震、光纤断 裂、收发元器件失效等;软故障是指传输过程中由于信号逐步衰减,而造成信息丢失的事 件,如光纤损耗増大等。

不难看出,硬故障对网络业务影响较大,但处理方便;软故障对网络业务的影响比较小,但出现机率较高,不容易被发现,并且难以对故障精确定位。

因而,从科研角度出发,光网中的故障按位置区分更利于研究:
\begin{itemize}
\item 信道故障是由于该信道对应的特定激光器或接收器出故障引起。
\item 链路故障主要是由于光纤断裂引起。
\item 节点故障主要是由于断电或是地震等灾难性情况引起。
\end{itemize}

网络故障主要表现为链路故障,链路故障恢复策略可分为主动式和被动式两种\cite{kvalbein2009multiple}。被动式策略在网络故障后动态自适应地进行全网资源重分配,但
路由重新收敛花费较多的时间而不可接受。因此目前故障快速恢复研究以主动式策略为主,通过提前对网络进行资源规划和预留,使得故障时能迅速切换,如基于多拓扑\cite{shand2010ip}和基于备份路径的故障恢复技术。多拓扑技术需要配置多个拓扑子层,路由存储消耗大;基于备份路径的故障恢复技术提供端到端路径重路由,在全局范围内进行流量分配,易于基于现有协议实现。因此,备份路径技术是当前故障恢复领域研究的热点\cite{yang2014keep,suchara2011network,banner2010designing}。

\section{网络故障恢复}
网络的生存性实现机制根据是否预留备用资源和是否进行重路由计算,通常把故障处 理策略分为保护(protection) 和恢复(restoration)两种。保护措施和恢复措施均是在 网络故障情况下,使受损的业务得以重新运行。原理上,两者均利用重路由方式(重新选 择新的路由来代替故障路由),继续故障业务数据传输。就具体实施方案而言,保护和恢 复方法又有所不同。两种方案各有利猝,恢复机制能够提高网络资源的利用效率;保护机制可提供更快的恢复时间,有利于网络QoS的保障。下面我们将做详细区分。

所谓保护方案是指,事先为业务分配好预留的保护资源即备份资源,主要利用节点 之间预留的备用资源来实现网络保护。当故障发生时,业务可以由事先预留的保护资源承 载,即将工作通路上的通信信号倒换到备用通路上,使工作信号通过预留的保护通路维持 业务正常传输;而恢复方案指,并不事先为业务分配预留的保护资源,在检测到故障时, 动态地从网络中寻找替代的路由,来承载受故障影响的业务。如果此时找不到合适的路由,那么,该工作路径上携带的业务就会丢失。

两种方案对比如下:
\begin{itemize}
  \item 由于保护机制是事先对特定的故障做了假设,进而预留了备份资源。它能够确保 对于预料之中的故障(如单链路失效〉业务的恢复,而对于些未知故障(如多链路同时 失效),则不能确保其恢复能力。
恢复机制是在发生故障后实时、动态地寻找可用资源,这种方式灵活性较强,W以针 对多种失效情况进行恢复,特别是对于多链路同时失效下的恢复。但由于事先没有预留备 份资源,因此在很多情况下,网络的资源状况不能确定,也就无法确保及时恢复。
  \item 保护机制是在牺牲资源利用率的前提下实施的,因而可以较快的实现业务恢复;恢复机制是动态地根据当前网络状态重路由,从而可以实现较高的资源利用率,但业务恢复 时间较长。
\end{itemize}
由上面的阐述可见,保护和恢复方案两者不存在互斥现象,因此,我们在进行网络的 生存性设计时,往往会考虑将保护和恢复方案折中的办法,取两种方案中的优点合并,从 而为客户提供多种服务级别,尽可能做到在恢复时间、运行保障与效率及成本之间取得平 衡。同时保证故障情况下,时效性和网络资源的合理化利用,即保证在一定的约束下为工 作通路预先选择好保护通路。这就相当于以网络生存性为约束条件的资源优化的问题。

一般而言,运营商根据不同的服务标准,对不同业务选择不同的方案,选择依据如下:
\begin{itemize}
  \item 对于重要的业务,如金融业,一般采取保护的方案。
  \item 对于一些特殊的拓扑,如链形、环形或环网相交等可以采用保护方案。
  \item 对于网络拓扑连通性强,且对网络的资源利用率要求较高的一些网络,则可以选
用合适的恢复方案。
\end{itemize}

由于现实中网络资源的不足,在实际的网络操作中,通常以保护机制为基础,來保 障一些可预料的故障,(如光缆断裂等公共失效故障〉,然后,再使用恢复机制进行加强,保障整网范围内的故障或失效。根据不同要求的业务可以选择不同类型的方法以保证其生存 性,例如对于实时业务,可以使用链路/节点保护,即预先建立保护通道和预留资源的保护 方式;而对于“尽力而为”的业务,就可以按需建立保护通道,或者依靠高层的恢复机制。 与恢复机制相比,保护机制具有更高的可靠性和更快速的恢复能力,这更适用于具有巨大 传输容量的光网络,因此本文主要研究的是保护机制。
\section{网络保护策略}
随着技术的发展,网络所承载的信息流量显著增加,特别是在主干网中,网络中设备 接口的速率达到2。 5Gb/s 以上,即使仅持续几秒的短暂性故障也会造成大量的数据丢失。 因此,为故障的业务寻找新的传输路由,并使其在尽可能短的时间内自愈变得越来越重要。 目前,许多文献对于光层故障保护方案进行了大量的研究,根据不同的功能或方式总结如下:
\begin{itemize}
  \item 从重路由的角度分为:基于路径保护、基于链路的保护及区段保护。
  \item 根据备用资源的预留方式可分为:专用保护和共享保护。
  \item 按照路由的计算方式可分为:预计算和实时计算。
\end{itemize}
\subsection{路径保护}
所谓通道(路径)保护M是指业务故障恢复由通道两端的终端节点来完成。具体说来, 基于通道的保护机制是指对工作路由事先预留一条备用保护路由,在故障发生后,用预留的保护通道来传输故障通道中的业务流,从而取代故障通道,实现业务重路由。在通道保 护方案中,每条光路在建立时就己经预设了一条端到端的备用通道并预留了备用资源。一 且网络中发生故障,受到影响的通道的源-目的节点对间的业务流将自动由工作通道切换 到这两个节点间的另一条与故障通道链路不相交的备用通道上来,从而保证业务恢复。即 当发生故障时,其切换过程只涉及源、目的节点,与中间节点无关,由于是源节点和目的 节点启动保护倒换,因而对故障的具体定位要求不高,但信令协议必须快速准确的将故障 消息传送至源、目的节点。工作状态如图2。 4所示:

通道保护又可分为专用通道保护(备用资源为某条工作通道专用)和共享通道保护(备 用资源能同时为多条工作通道提供保护)。专用保护通常是指1+1通路保护和1: 1通路 保护。共享保护是指1: N保护方式。
\subsection{链路保护}
基于链路的故障保护方案,是指业务请求经过的每一条链路,都有一条保护路径对其 进行保护,一旦链路出现故障,业务将越过故障点,直接转到保护路径上。它是通过处理 与故障点相邻的节点来实现对业务的恢复的,是将受故障影响的业务流绕过故障链路来进 行重路由。即:在与故障点邻接的两点间,为该故障链路寻找一条可绕过该故障点的备用 路由。显然,该保护方案中,参与保护切换的节点数较少,因而具有较强的本地性,恢复 速度较快。同时由于它的本地性,也使得资源浪费过多,从而,无法有效的利用资源。
在该方案中,对于不同链路中的业务,只要不同时发生故障,就可以共享相同的保护 路径,因此也可以分为专用链路保护和共享链路保护两种。前者是指对于某一链路,提 供专门的保护路径,其他链路则不得使用该专用路径;而后者允许不同的链路保护路径在 其重叠的部分实现共享。比较而言,后者比前者资源利用率要高,而前者较后者的保护力 度大。
\subsection{区段保护}
区段保护是折中了通路保护和链路保护各自的特点而得到的一种中庸的保护方式。区 段保护是指在一对节点之间出现光纤断裂故障时,则该段链路中的业务被倒换到这两个节点之间的另一根光纤中。如果两个相邻节点之间的一根光纤发生断裂,则类似于链路保护。
\subsection{本章小结}
网络生存性设计的目的是提高网络的健壮性,由于网络故障不可避免,那么故障后的 及时修复成为网络性能的一个重要方面。本章给出了故障业务恢复方法的综述,比较得出共享保护方案优于专用保护方案,同时通道保护方案优于链路保护方案,由此提出,一般 共享通道保护方案的性能较高。



%在满足delay 约束的同时达到两条路径总的
%花费(cost)最小。当给定的delay 约束针对两条路径的端到端延时总和时,问题被称为DCLDOP-I(delay
%constrained link disjoint optimal paths),当给定的delay 约束针对路径对中每条路径的端到端延时时,问题被称为
%DCLDOP-II。文献[Researches on the problem of link disjoint paths with QoS constraints]对DCLDOP-I 和DCLDOP-II 问题进行了建模,证明了这两种问题同属于NP 完全问题。文献
%[4]针对DCLDOP-I 问题提出了两种近似求解算法。文献[Constrained shortest link-disjoint paths selection: A network programming based approach]研究了总延时受限下的k 条cost 最小链路分离路径
%问题。文献[On the complexity of and algorithms for finding the shortest path with a disjoint] 提出了Min-Min 问题,旨在求解两条满足QoS 约束的分离路径且满足较短的路径cost 最小。文献[Link-Disjoint shortest-delay path-pair computation algorithms for shared mesh restoration networks]
%通过求解总延时最小的链路分离路径对来解决单链路失效后的路由恢复问题。

% !Mode:: "TeX:UTF-8"
\section{相关的研究及发展趋势}
\subsection{共享风险链路组完全不相交路径对问题}
网络故障主要表现为链路故障,链路故障恢复策略可分为主动式和被动式两种\cite{kvalbein2006fast,qi2012research}。被动式策略在网络故障后自适应动态地进行全网资源重分配,但路由重新收敛花费较多的时间而不可接受。因此目前故障快速恢复研究以主动式策略为主,通过提前对网络进行资源规划和预留,使得故障时能迅速切换,如基于备份路径和基于多拓扑\cite{shand2010ip}的故障恢复技术。基于备份路径的故障恢复技术提供端到端路径重路由,在全局范围内进行流量分配,易于基于现有协议实现;多拓扑技术需要配置多个拓扑子层,路由存储消耗大。因此,备份路径技术是当前故障恢复领域研究的热点。

网络服务质量需求是多约束不同属性的条件,可以对这些条件加权值和归一化统一成单一约束条件,网络大部分工作流量集中主路径上,备份路径是当主路径出现故障时,临时使用来恢复故障的路径,所以对备份路径的服务质量需求并不需要特别苛刻,而应该尽量提高主路径的服务质量需求。而网络故障一般为链路故障,SDN/NFV架构下多租户数据中心Overlay 网络和传输层光网络的故已经不是单独一条链路的故障,而是拓扑图逻辑上几条链路同时故障,此时延伸到风险链路组SRLG的故障。

共享风险链路组(SRLG)是一组链路共享相同的一个组件,该组件的故障会导致在这个组里所有链路同时发生故障。就路径保护而言,尽管某些链路或者节点不相交路径算法已经提出来,SRLG不相交路径问题是比较棘手的,而且这些原有研究是限制在一定领域的。比如当每个SRLG 只包含一条链路时,这个SRLG不相交路由问题可以简化为链路不相交路径问题,而通过节点分裂方法(node split method)\cite{ford2015flows}节点不相交路径问题可以转化为链路不相交路径问题。因为SRLG组通常包括的链路超过一条,并且网络中的链路通常可以属于多个SRLG组里,以至于求一对SRLG 不相交路径问题比求一对链路或者节点不相交路径问题要困难得多。

文献\mycite{rostami2007cose,todimala2004imsh,xu2003trap}是获得完全不相交的SRLG路径对的启发式算法。一种链路不相交算法被扩展到SRLG不相交算法\cite{rostami2007cose},从而产生了冲突的SRLG冲突集算法。该算法将SRLG 不相交路径问题转化为Min-Min问题。文献\mycite{gomes2010obtaining}提出了一种新的CoSE启发式算法,用于求解CoSE不相交的最小和问题。文献\mycite{todimala2004imsh}的SRLG不相交路径问题也被描述为一个Min-Sum问题,并通过迭代启发式来求解,即迭代修正的Suurballe启发式(IMSH)。文献\mycite{xu2003trap}提出了一种陷阱避免方案TA算法 ,其中考虑到的算法试图避免陷入陷阱(即算法无法找到SRLG不相交路径对的情况)。文献\mycite{luo2007insights}提及的段保护用于避免陷阱。文献\mycite{oki2002disjoint}考虑了一种基于k-最短路径的算法(加权SRLG 或WSRLG),其中根据与链路相关的SRLG 成员数将费用分配给链路。文中还考虑了根据链路与其他链路分担的风险分配链路成本的思想,其中提出了一种主动SRLG-多路径选择(ASPS)算法,该算法允许提高资源利用率(因为带宽资源在备用路径之间共享),并考虑到不同用户的区分可靠性要求。文献\mycite{pan2006heuristics}中也考虑到共享资源保护的背景,文中提出了一种混合保护方案,该方案将共享路径保护和共享链路保护联合应用于具有风险共享链路组的WDM网络中。文献\mycite{wang2007impairment}提出了两种不同的算法来计算给定的原点对和目标对的最优路径对,在这种情况下,路径对的最优性可以用两种不同的方式定义,从而导致两个不同版本的缺陷感知最佳路径对问题。文献\mycite{cheng2007multiple}中考虑了一个不同的上下文,其中作者试图从一个预计算的路由表中找到一个AP和几个BP,并以最小的总附加带宽来承载特定的连接。文献\mycite{rostami2007cose}提出了一种改进型CoSE 的变体-modified-CoSE,在该算法中,如果不存在SRLG不相交路径对,则使用的SRLG路径对(其中AP最短且BP共享较少的SRLG)。文献\mycite{silva2011heuristic}提出了TA算法\cite{xu2003trap}的一个变体指定的TA-Max。在TA-Max中迭代地考虑候选AP(为了增加成本),并为每个候选AP计算一个具有最小成本的BP,最后的解决方案是共享最少数目的公共SRLG的路径对。


现如今研究SRLG完全不相交路径问题的精确算法只有\cite{rostami2007cose,hu2003diverse,todimala2004imsh}。算法\mycite{rostami2007cose}提出一种基于SRLG冲突集的分而治之方法,其分拆的子问题数规模,为得到最优解需消耗大量时间。算法\mycite{hu2003diverse} 提出SRLG完全不相交问题的线性规划方程解法。算法\mycite{todimala2004imsh}求解的不相交路径对,其优化的目标是路径对的权重和最小化。而其他研究SRLG 不相交路径问题的算法\mycite{xu2003trap} 都是启发式的非精确算法,即求解的路径对可能非最优路径对。算法\mycite{todimala2004imsh} 获得的路径对的优化目标是Min-Sum,即两条两路的权重和最小化,不保证主路径的权重最小化。在实际网络上,主路径承担主要的业务就需要其路径权重最小化。综上,求一对Min-Min SRLG完全不相交满足服务质量需求的路径对问题是现今网络故障可生存性算法的研究重点。

%故障恢复的本质在于维持故障链路承载流量的传输,因此应从流量持续传输角度解决故障恢复问题。大部分故障恢复工作集中在如何选择可靠备份路径上,且一般利用单一备份路径进行故障恢复。然而当流量超出备份路径可用带宽时,单一备份路径无法满足故障恢复的要求。受到这一启发,文献\cite{wang2010r3}将多路径技术引入到故障恢复中,采用多条备份路径共同承担流量,减少了流量的丢弃。在此基础上,文献\cite{banner2010designing}考虑了不同故障状况,通过将重路由流量分配给有跳数限制的多条备份路径,在网络投入运行之前就设计好应对各种故障场景的最低容量备份网络。文献\cite{suchara2011network}提出一种结合故障恢复与流量工程的网络结构,在多路径故障恢复基础上进行流量工程优化,其目的是进行负载均衡,且假设备份路径可用带宽满足故障恢复需求。文献\cite{zheng2014cross}提出一种跨层故障恢复模型,考虑了备份链路的可靠性,提升了故障恢复成功率。然而,上述故障恢复算法大都以最大化重路由为目标,未考虑恢复后的流量是否满足用户的需求。但是经由备份路径的重路由流量即使最终传输成功,由于链路过载、时延超时等原因,也无法满足业务的服务质量需求(Quality of Service,QoS),属于无效流量。虽然文献\cite{misra2009polynomial}提出了一种满足QoS 约束的自适应调整的多路径路由,但未考虑故障恢复问题。因此,目前已提出的大部分链路故障恢复算法不能很好地确保业务的服务质量。


\subsection{可生存性虚拟网络嵌入问题}
在考虑虚拟网络的嵌入过程中,常见的算法主要由两类:首先第一种是节点映射和链路映射彼此独立进行处理,将虚拟节点映射到物理节点上,然后由于此时已经映射的节点在底层拓扑的位置是己知的,因此接下来的链路映射就可以概述为多商品流问题\cite{even1975complexity}(Multi-commodity Flow Problem,MCF),这种算法比较简单,但是由于节点和链路 映射是独立进行的,只考虑到局部最优解无法获得全局最优解:第二种是节点和链路映射同时考虑,被称为虚拟网络嵌入协调过程,虽然它们都考虑到了全局情况,但是由于问题本身原理的限制,问题的复杂度规模是NP-hard,存在着成本比较高和结果不够精确的缺点。在虚拟网络的嵌入算法中,一般会考虑动态映射和静态映射两种情况\cite{fischer2013virtual},有效的嵌入算法在降低成本,提高资源使用率这一方面就显得尤为重要。



现有基于物理节点故障恢复的可生存性虚拟网络嵌入算法主要采用主动保护\cite{yu2011cost} 和被动恢复\cite{rahman2010survivable}两类策略。主动保护策略在进行虚拟网络请求映射前为虚拟网络预置备份资源以供后续故障恢复使用,被动恢复策略在进行虚拟网络请求映射前不预置备份资源,在后续故障发生后再使用节点迁移或者现有的物理网络资源进行故障恢复。采用主动保护策略的算法包括FI-EVN\cite{yu2011cost}、FD-EVN\cite{wang2014survivable} LC-SVNE\cite{hu2012location}等,采用被动恢复策略的算法包括NB-SVNE\cite{bo2014dynamic}、MR-SVNE\cite{qiang2014heuristic}等。

可生存性虚拟网络嵌入算法的研究从基于主动保护和被动恢复两个方面提出了相应的故障恢复方法,但是依旧存在下面的一些不足:
\begin{enumerate}
  \item 现有基于传统网络的SVNE算法中的主动保护策略比较固化,在物理网络无法为VNR提供全部所需备份资源的时候,会拒绝该VNR, 进而导致虚拟网络请求接受率和网络运营收益偏低,而SVNE算法中的被动保护策略易造成故障恢复效率低和故障恢复成功率抖动大的情况。
  \item 现有基于SDN网络的SVNE相关算法中采用的也是主动保护策略, 也存在主动保护策略的请求接受率和网络运营收益低的问题。虽然算法在此基础上增加了备份资源重分配机制\cite{yu2008rethinking}和虚拟网络映射资源来增网络运营收益加和网络资源利用率,但是资源动态调整机制不仅会增 加算法的时间复杂度,还易造成虚拟网络抖动,影响虚拟网络服务质 量。考虑到算法时间复杂度问题,本文暂不考虑使用资源重映射机制。
\end{enumerate}

综上所述,目前的研究对于虚拟网络的保护策略不够灵活,或多或少都会出现性能短板问题,如主动保护策略中的虚拟网络请求接受率和收益低,被动恢复策略故障恢复效率低,传统网络下的SVNE算法无法对物理资源进行集中管控, 需要在物理设备间进行大量的通信,这会带来大量的通信开销,因此提出一个根据物理网络资源现状灵活实现虚拟网络冗佘备份的可生存性虚拟网络嵌入算法,可以来提高虚拟网络请求接受率、故障恢复效率和物理网络资源利用率,这非常有意义。

因此,为提高故障恢复成功率和虚拟网络请求接受率,使运营商的物理网络能够为租户提供更为可靠的虚拟网络服务,本文提出了星型分解动态规划节点映射的可生存性虚拟网络嵌入算法,此算法可以应用在主动保护和被动恢复两种策略的情形,与现有算法的比较,具有更高的可生存性接受率,资源利用率和收益,并且故障恢复效率更高,并且能拓展到应有于地域受限的虚拟网络嵌入问题和物理网络多节点故障的情形。




%为此,本文针对QoS 约束下的链路故障恢复问
%题进行研究,即在网络可用带宽和业务时延需求约
%束下进行最大化重路由流量问题求解。首先基于多
%备份路径策略建立概率关联故障模型和重路由流量
%丢弃优化目标,并构建QoS 约束的故障多备份路径
%恢复问题的数学优化模型。然后,设计QoS 约束的
%链路故障多备份路径恢复算法(Multiple backup
%Paths Recovery for link failure with QoS constrain
%algorithm, MPR-QoS) 对此问题进行求解。
%MPR-QoS 算法在构建单条备份路径时,利用改进
%的QoS 约束的k 最短路径法进行拼接,并以最大化p
%减少重路由流量丢弃为目标,且分配给高优先级链
%路更多的保护资源。此外还证明了算法的正确性并
%对时间空间复杂度进行了分析。最后,在NS2 仿真
%环境下从故障恢复率、重路由流量QoS 满足率、链
%路过载率和算法运行时间等方面验证了本文算法的
%优越性。

% !Mode:: "TeX:UTF-8"
\section{论文的主要研究内容和组织结构}
\subsection{论文的主要研究内容}
SDN作为新型网络架构,NFV作为未来网络研究的重要领域,两者的结合具有新的研究意义。在网络可生存性领域中,当网络故障发生时对业务主路径提供备份路径是一个重要基础研究领域,在网络虚拟化的过程中,不可避开的一个问题就是虚拟网络嵌入算法的研究。
\subsubsection{Min-Min SRLG 完全不相交路径对问题}
基于图论寻找SRLG完全不相交路径时,对陷阱问题提供了新的见解。具体来说,对于带有陷阱问题的AP,我们观察到AP路径上存在一组链路集,任何通过所有这些“问题”链路AP都不能找到一个SRLG完全不相交的BP。我们称之为SRLG冲突链路集。一旦遇到陷阱问题,建议寻找SRLG冲突链路集,而不是搜索所有可能的替代路径,基于最大流最小割定理\cite{ford2015flows}。 进一步提出了一种分而治之的min-min SRLG完全不相交路由算法,将原路由问题划分为多个子问题并行执行,找到可行的AP和BP对。研究的主要内容如下:
\begin{itemize}
  \item 通过巧妙地设置链路容量来构造一个新的流图,以便于发现SRLG 冲突链路集。
  \item 提出了一种寻找最小SRLG冲突链路集的算法,这有助于减少搜索替代SRLG完全不相交路径对的复杂性。
  \item 根据SRLG的风险共享的特性,将SRLG冲突链路集最小查找问题转化为子集覆盖问题,使我们能够应用通用算法在不同的复杂SRLG场景(包括一个或多个SRLG模式的链路)中寻找最小SRLG冲突链路集。
  \item 提出了一种新的分而治之算法,该算法能在遇到陷阱问题时将原有的min-min SRLG完全不相交路由问题划分为多个并行执行子问题。与现有技术相比,这样的解决方案搜索过程可以利用现有的AP搜索结果和并行执行来加快的路径查找。
  \item 在一个多核CPU平台上进行了广泛的仿真,以评估所提出的算法。仿真结果表明,该算法能够以更快的速度在不同的网络场景中找到最佳解。
\end{itemize}

\subsubsection{单物理节点故障可生存性虚拟网络嵌入问题}
嵌入算法主要考虑的是虚拟网络节点和链路向底层物理网的节点和链路的映射,考虑如何为虚拟网络在底层物理网中找到满足条件的最优嵌入,并且尽可能地提髙网络资源的利用率。%鉴于此,本文做了如下研究:
%\begin{enumerate}
%  \item 首先,研究SDN虚拟网络技术,对SDN网络虚拟化的环境进行了研究,并且探讨和总结了虚拟网络向底层物理进行嵌入时的不同嵌入算法。
%  \item 在虚拟网络向底层物理网络映射时,考虑节点和链路的映射以及路由算法的设计,在嵌入算法的设计中,假设节点己经映射并且节点具有功能类型的约束。分别设计算法实现此时提供可生存性的虚拟网络嵌入算法。
%\end{enumerate}

本课题将结合位置约束的概念,以虚拟网络请求的接受率、物理网络资源利用率和故障恢复率等指标作为算法性能评价指标,提出一个星型分解动态规划节点映射的可生存性虚拟网络嵌入算法。为了实现课题目标,需要进行以下研究工作:
\begin{enumerate}
  \item 调研现有的可生存性虚拟网络映射算法,以及基于SDN的虚拟网络映射算法和可生存性虚拟网络映射算法,分析不同算法的优缺点,以及存在的共性问题。
  \item 提出一种星型分解动态规划节点映射的可生存性虚拟网络嵌入算法,提出虚拟星型图和物理星型图的概念,给出虚拟星型图与物理星型图之间节点映射的权值设定,转换节点映射的问题为多背包问题,通过动态规划方法解决这个多背包问题,理论分析了动态规划方法时空间复杂度。
  \item 设计和搭建仿真平台,对提出的嵌入算法进行仿真,并实现相关的对比方案。对实验结果进行分析,评估本文方案和对比方案的性能优劣得出,提出的算法比其他算法在接受率和资源利用率等指标上结果更优。
\end{enumerate}

\subsection{论文的组织结构}
本文依据网络可生存性、不相交路径以及虚拟网络嵌入等方面的研究现状与发展趋势,并结合自身对以上的探索和研究,将本文分为五章,各章主要内容如下,其由于硕士论文篇幅限制,本文有关不相交路径算法、虚拟网路嵌入算法和可生存性虚拟网络嵌入算法的总结将不在本文中,可在github网址\cite{Thesis}得到论文更详细版本:
\begin{itemize}
  \item 第一章为绪论部分,主要介绍了本课题的研究背景及其意义并且简要地分析了研究现状和发展趋势。
  \item 第二章介绍了网络可生存性相关原理,概述了网络的常见故障和网络故障恢复的两大机制,接着详细介绍了网络故障保护策略。
%  \item 第三章介绍了不相交路径的相关技术,概述了不同类型的不相交路径的问题,然后给出不同类型的不相交路径的约束条件,实际应用和时间复杂度的总结。
  \item 第三章主要是Min-Min SRLG完全不相交路由算法的设计与实现,在存在陷阱问题的情况下,提出了一种求解Min-Min SRLG完全不相交路由问题的有效算法。为了降低搜索复杂性,提出了一种分而治之的解决方案,将原Min-Min风险共享链路组不相交路由问题划分为多个子问题,该子问题基于从AP路径遇到陷阱问题导出的SRLG冲突链路集,在一个多核CPU平台上使用拓扑上进行了广泛的仿真。
  %\item 第五章介绍了网络虚拟化的相关技术,概述了网络虚拟化环境和网络虚拟化技术特征,接着对虚拟网络嵌入算法分类,总结了不同虚拟网络嵌入算法的优化目的,是否节点/链路映射协调和主要贡献,然后总结了可生存性的虚拟网络嵌入算法故障类型,优化目标和处理机制,最后给出虚拟网络嵌入算法常用的评价度量。
  \item 第四章主要是单物理网络节点故障可生存性虚拟网络嵌入算法的设计与实现,当一个虚拟请求已经嵌入和运行在底层物理网络中时,突然一个底层物理节点随机独立的出现故障,提出的星型分解动态规划节点映射的可生存性虚拟网络嵌入算法可以预先分配备用资源来预防故障失效,进行多种算法指标度量的仿真。
  \item 第五章对现有的工作和研究成果进行了总结,并进一步对其中的某些技术问题探讨了未来研究的方向。
\end{itemize}

%% !Mode:: "TeX:UTF-8"
\section{预备知识}


\subsection{计算理论}


%绪论”的内容至少应该包括选题的价值与意义、文献评论、本文的思路、资料和方法、各章节的主要内容及逻辑安排等,以此彰显本项研究与已有成果之差异,强调本项研究在资料、方法上的独特性,以及全文写作的基本思路,以俾读者更好地把握全文,并激起阅读的兴趣。

%该如何进行文献评论呢?在我们看来,文献评论指对所涉文献的理解、吸收与批判,就内容而言,主要包括资料、方法、理论和总体四个方面,兹分别叙述如下。
%
%其一,理解并学习他人如何获取资料,判别资料的价值并思考进一步开掘新资料的可能性。
%
%其二,将已有各种研究方法进行认真的评估,判断各种研究方法之优劣并思考采用新方法的可能。
%
%其三,将已有的各种理论置于不同的概念架构中进行分析,并形成自己新假设。或者辨识已有概念间的前提假设,提出对某行为或现象的可能解释,形成新概念。
%
%其四,把握有关研究的进展,探讨改进的可能性或思考未来的研究是否更有意义,是否能得出更为显著的结果,即是否获得学术上的新进展。  