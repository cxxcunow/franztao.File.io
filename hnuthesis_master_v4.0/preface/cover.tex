% !Mode:: "TeX:UTF-8"

\chnunumer{10532}
\chnuname{湖南大学}
\cclassnumber{TP391}
%\cnumber{**}
\cnumber{S151000891}
\csecret{普通}
\cmajor{计算机网络}
\cheading{硕士学位论文}      % 设置正文的页眉,以及自己的学位级别
\ctitle{SDN/NFV网络架构可生存性算法研究}  %封面用论文标题,自己可手动断行
\etitle{The Research on SDN/NFV Network Architecture Survivable Algorithm }
\caffil{信息科学与工程学院} %学院名称
\csubjecttitle{学科专业}
\csubject{计算机科学与技术}   %专业
\cauthortitle{研究生}     % 学位
\cauthor{陶恒}   %学生姓名
%\cauthor{**}   %学生姓名
\ename{Heng~~TAO}
%\ename{*~~*}
\cbe{B.E.~(Hunan University of Science and Technology)~2015}
%\cms{M.S.~(Hunan University)~2018}
\cdegree{thesis}
\cclass{Master of engineering}
\emajor{Computer Science and Technology}
\ehnu{Hunan~University}
%\esupervisor{**}
\esupervisor{Kun Xie}
\csupervisortitle{指导教师}
\csupervisor{谢鲲~教授} %导师姓名
%\csupervisor{**~教授} %导师姓名
\elevel{Professor} %导师职称
\cchair{邝继顺~教授}%邝继顺
%\cchair{**}
\ddate{~~~~2018年~~~5月~~~20日}
\edate{May,~2018}
\untitle{湖~~南~~大~~学}
\declaretitle{学位论文原创性声明}
\declarecontent{
本人郑重声明:所呈交的论文是本人在导师的指导下独立进行研究所取得的研究成果。除了文中特别加以标注引用的内容外,本论文不包含任何其他个人或集体已经发表或撰写的成果作品。对本文的研究做出重要贡献的个人和集体,均已在文中以明确方式标明。本人完全意识到本声明的法律后果由本人承担。
}
\authorizationtitle{学位论文版权使用授权书}
\authorizationcontent{
本学位论文作者完全了解学校有关保留、使用学位论文的规定,同意学校保留并向国家有关部门或机构送交论文的复印件和电子版,允许论文被查阅和借阅。本人授权湖南大学可以将本学位论文的全部或部分内容编入有关数据库进行检索,可以采用影印、缩印或扫描等复制手段保存和汇编本学位论文。
}
\authorizationadd{本学位论文属于}
\authorsigncap{作者签名:}
\supervisorsigncap{导师签名:}
\signdatecap{签字日期:}


%\cdate{\CJKdigits{\the\year} 年\CJKnumber{\the\month} 月 \CJKnumber{\the\day} 日}
% 如需改成二零一二年四月二十五日的格式,可以直接输入,即如下所示
% \cdate{二零一二年四月二十五日}
\cdate{~~~~2018年~~~5月~~~8日} % 此日期显示格式为阿拉伯数字 如2012 年4 月25 日
\cabstract{
现如今网络整体规模和网络功能类型的复杂度越来越大,新技术对网络性能需求不断提高,传统的网络架构己经无法满足按需调动、快速配置等需求,软件定义网络(SDN)的提出就是为了应对现有网络架构无法解决的问题和突破网络性能瓶颈的限制。利用网络功能虚拟化(NFV)技术将网络资源进行虚拟化,在很大程度上可以解决现有网络无法解决的问题和到达现有网络无法达到的网络性能需求。网络服务运行在实际的物理设备上,这将必然产生不可避免的网络故障,SDN/NFV新型的网络架构在网络的可生存性领域比传统网络架构更加高效的检测、处理和恢复网络故障。

光网络和Overlay网络中为了保证网络的可生存性,基于SDN的架构能高效实现不相交路径路由功能,快速的在源节点和目的节点之间寻找满足一定QoS约束的不相交路由。当主用路径出现故障时,将其承载的业务流转换到备用路径上,从而实现快速的业务恢复,因此快速不相交路径算法的研究如今具有很高的研究价值。网络虚拟化的过程中,如何为虚拟网络在底层物理网中找到满足约束条件且最优的嵌入,并且提供可生存性保护需求,来保证底层物理网资源失效情况下原本虚拟网络的业务正常运行,可生存性虚拟网络嵌入问题的研究已经是NFV研究中的重要邻域。
%SDN作为新型网络架构,网络虚拟化作为未来网络研究的重要领域,两者的结合具有很高的研究意义。传统分布式网络通过复杂的MPLS协议来实现对源节点和目的节点之间提供多条QoS路径,以提高网络的可生存性,这一做法已不能满足在网络连接出现故障时保持业务高效灵活迅速的提供可生存性保护需求。嵌入算法主要是考虑虚拟网络节点和链路向底层物理网络的映射,

本文结合当今研究热点,基于SDN/NFV网络架构下以可生存性算法作为课题的研究方向,主要研究了以下三个方面的内容:

研究网络可生存性技术,对网络故障失效环境进行了研究,并且探讨了已有各种情形的路径保护算法和不相交路径算法。

研究共享风险链路组不相交的约束条件,提出冲突边集合的概念,通过容量设置来获得冲突边集合,以冲突边集合来分而治之原问题,并行解决SRLG不相交路由问题,并且与己有算法进行理论分析和实验对比,结果表明我提出的算法优于现有其它算法。

%其次,研究SDN虚拟网络技术,对SDN网络虚拟化的环境进行了研究,并且探讨了虚拟网络向底层物理网进行映射时的不同嵌入算法和可生存性嵌入算法。

研究节点和链路映射以及可生存性需求之间的关系,在可生存性虚拟网络嵌入算法的设计中,考虑到节点带有特定功能约束条件,本文结合已有算法的不足,提出了星型分解动态规划节点映射的可生存性虚拟网络嵌入算法,并且与己有算法进行实验对比,实验结果表明我提出的算法优于现有其它算法。
}
%中文摘要应将学位论文的内容要点简短明了地表达出来,约500~800字左右(限一页),字体为宋体小四号。内容应包括工作目的、研究方法、成果和结论。要突出本论文的创新点,语言力求精炼。为了便于文献检索,应在本页下方另起一行注明论文的关键词(3-7个)。
\ckeywords{软件定义网络;~~网络功能虚拟化;~~不相交路径;~~虚拟网络嵌入;~~ 可生存性;~~风险共享链路组}
\eabstract{Nowadays, the complexity of the whole network scale and the network function type is more and more large, the new technology is increasing the demand for network performance, and the traditional network architecture has been unable to meet the needs of on-demand mobilization, rapid configuration, and so on. Software defined Network (SDN) is proposed to deal with the existing network architecture can not solve the problem and break through the network performance bottlenecks. Using network function virtualization (NFV) technology to virtualize network resources can to a large extent solve the problems that the existing network cannot solve and reach the network performance requirements that the existing network cannot achieve. Network services run on actual physical devices, which will inevitably lead to inevitable network failures. SDN / NFV new network architecture can detect, process and recover network failures more efficiently than traditional network architecture in the field of network survivability.

In order to ensure the survivability of optical network and Overlay network, the architecture based on SDN can efficiently realize disjoint path routing function, and find disjoint routing between source node and destination node quickly. When the main path fails, the traffic flow is converted to the standby path, so the fast disjoint path algorithm has a high research value. In the process of network virtualization, how to find the best embedding in the underlying physical network, and how to provide survivability protection for the virtual network. To ensure the normal operation of the original virtual network under the condition of the failure of the underlying physical network resources, the research of the survivability virtual network embedding problem has become an important neighborhood in the research of NFV.

In this paper, based on the research direction of survivability algorithm under the SDN/NFV network architecture, we mainly study the following three aspects:

In this paper, the survivability technology of network is studied, and the fault failure environment of network is studied, and the existing path protection algorithms and disjoint path algorithms are discussed.  

In this paper, the constraint conditions of disjoint shared risk link group are studied, and the concept of conflict edge set is proposed. The conflict edge set is obtained by capacity setting, the original problem of dividing and dominating the conflict edge set is obtained, and the SRLG disjoint routing problem is solved in parallel. The theoretical analysis and experimental results show that the proposed algorithm is superior to other existing algorithms.  

This paper studies the relationship between node, link mapping and survivability requirements. In the design of survivability virtual network embedding algorithm, considering that nodes have specific functional constraints, this paper combines the shortcomings of existing algorithms. A survivabal virtual network embedding algorithm based on star decomposition dynamic programming node mapping is proposed and compared with the existing algorithms. The experimental results show that the proposed algorithm is superior to other existing algorithms.

}
\ekeywords{Software Defined Network(SDN);~~Network Function Virtualization(NFV);~~Disjoint Path;~~ Virtual Network Embedding(VNE);~~Survivability;~~Shared Risk Link Group(SRLG)}
\makecover
\clearpage
