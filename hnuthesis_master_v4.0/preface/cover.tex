% !Mode:: "TeX:UTF-8"

\chnunumer{10532}
\chnuname{湖南大学}
\cclassnumber{TP391}
\cnumber{S151000891}
\csecret{普通}
\cmajor{计算机网络}
\cheading{硕士学位论文}      % 设置正文的页眉,以及自己的学位级别
\ctitle{SDN网络的可靠路由算法研究}  %封面用论文标题,自己可手动断行
\etitle{Based on SDN Disjoint Path Algorithm Rearch}
\caffil{信息科学与工程学院} %学院名称
\csubjecttitle{学科专业}
\csubject{计算机科学与技术}   %专业
\cauthortitle{研究生}     % 学位
\cauthor{陶恒}   %学生姓名
\ename{Tao~~Heng}
\cbe{B.E.~(Hunan University)~2011}
%\cms{M.S.~(Hunan University)2010}
\cdegree{thesis}
\cclass{Master of engineering}
\emajor{Computer Science and Technology}
\ehnu{Hunan~University}
\esupervisor{Wang Wu}
\csupervisortitle{指导教师}
\csupervisor{谢鲲~教授} %导师姓名
\elevel{Professor} %导师职称
\cchair{~~~~~~~~}
\ddate{~~~~~~~~年~~~~月~~~~日}
\edate{April,~2018}
\untitle{湖~~南~~大~~学}
\declaretitle{学位论文原创性声明}
\declarecontent{
本人郑重声明:所呈交的论文是本人在导师的指导下独立进行研究所取得的研究成果。除了文中特别加以标注引用的内容外,本论文不包含任何其他个人或集体已经发表或撰写的成果作品。对本文的研究做出重要贡献的个人和集体,均已在文中以明确方式标明。本人完全意识到本声明的法律后果由本人承担。
}
\authorizationtitle{学位论文版权使用授权书}
\authorizationcontent{
本学位论文作者完全了解学校有关保留、使用学位论文的规定,同意学校保留并向国家有关部门或机构送交论文的复印件和电子版,允许论文被查阅和借阅。本人授权湖南大学可以将本学位论文的全部或部分内容编入有关数据库进行检索,可以采用影印、缩印或扫描等复制手段保存和汇编本学位论文。
}
\authorizationadd{本学位论文属于}
\authorsigncap{作者签名:}
\supervisorsigncap{导师签名:}
\signdatecap{签字日期:}


%\cdate{\CJKdigits{\the\year} 年\CJKnumber{\the\month} 月 \CJKnumber{\the\day} 日}
% 如需改成二零一二年四月二十五日的格式,可以直接输入,即如下所示
% \cdate{二零一二年四月二十五日}
\cdate{~~~~~~~~年~~~~月~~~~日} % 此日期显示格式为阿拉伯数字 如2012年4月25日
\cabstract{传统的QoS路由算法只在源节点和目的节点之间提供一条QoS路径,这一做法已不能满足在网络连接出现
故障时保持业务持续不间断地进行这一要求。分离路径算法试图在源节点和目的节点之间寻找满足一定QoS约束的分离
路径(链路分离或节点分离),一条主用路径,另一条备用路径。当主用路径出现故障时,将其承载的业务流转换到备用路径
上,从而实现快速的业务恢复。因此,分离路径算法研究有很重要的实用价值。
}
%中文摘要应将学位论文的内容要点简短明了地表达出来,约500~800字左右(限一页),字体为宋体小四号。内容应包括工作目的、研究方法、成果和结论。要突出本论文的创新点,语言力求精炼。为了便于文献检索,应在本页下方另起一行注明论文的关键词(3-7个)。
\ckeywords{软件定义网络;~~分离路径}
\eabstract{

Ensuring transmission survivability is a crucial problem for high-speed networks. Path protection is a fast and capacity-efficient approach for increasing the availability of end-to-end connections. The emerging SDN infrastructure makes it feasible to provide diversity routing in a practical network. For more robust path protection, it is desirable to provide an alternative path that does not share any risk resource with the active path. We consider finding the SRLG-Disjoint paths, where a Shared Risk Link Group (SRLG) is a group of network links that share a common physical resource whose failure will cause the failure of all links of the group. Since the traffic is carried on the active path most of time, it is useful that the weight of the shorter path of the disjoint path pair is minimized, and we call it  Min-Min SRLG-Disjoint routing problem. We prove this problem is NP-complete. The key issue faced by SRLG-Disjoint routing is the trap problem, where the SRLG-disjoint backup path (BP) can not be found after an active path (AP) is decided. Based on the min-cut of the graph, we  design an  efficient algorithm that can  take advantage of existing search results to quickly look for the SRLG-Disjoint path pair. Our performance studies demonstrate that our algorithm can outperform other approaches with a higher routing performance while also at a much faster speed.}
\ekeywords{Software Defined Network;~~Disjoint Path;~~ Quality of Service}
\makecover
\clearpage
