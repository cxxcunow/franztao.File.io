
\documentclass[10pt,journal,letterpaper]{IEEEtran1}
%\usepackage{latex8}
\usepackage{times}
\usepackage{amssymb}
\usepackage{amsmath}
\usepackage{graphicx}
\usepackage{epstopdf}
\usepackage{subfigure}
\usepackage[nocompress]{cite}
\usepackage{algorithm}
\usepackage{algorithmic}
\usepackage{titlesec}
\usepackage{multirow}
\usepackage{tabularx}
\usepackage{booktabs}
\usepackage{threeparttable}
\usepackage[normalem]{ulem}
\usepackage{url}
\usepackage{CJK}
%\usepackage{amsthm}
\pagestyle{plain}
\usepackage{geometry}
\usepackage{tikz}
\geometry{left=1.5cm,right=1.5cm,top=1.5cm,bottom=1.55cm}


%%\geometry{left=1.40cm,right=1.40cm,top=1.33cm,bottom=2.1cm}
%%\linespread{0.986}
%\setlength{\abovedisplayskip}{1 mm}
%\setlength{\belowdisplayskip}{0.5 mm}
%%\setlength{\abovecaptionskip}{0.0pt}
%%\setlength{\belowcaptionskip}{0.0pt}
%%\titlespacing{\section}{0pt}{4pt}{0 pt}
%%\titlespacing{\subsection}{0pt}{2pt}{0pt}
%%\titlespacing{\subsubsection}{0pt}{2pt}{0pt}
\usepackage[colorlinks,linkcolor=red,anchorcolor=blue]{hyperref}

\newcommand{\songti}{\CJKfamily{song}}

\newcommand{\rev}[1]{\uwave{#1}}
%\newcommand{\rev}[1]{#1}
%\renewcommand{\algorithmicrequire}{\textbf{Input:}}
%\renewcommand{\algorithmicensure}{\textbf{Output:}}

\newcommand{\del}[1]{\sout{#1}}  %revise the text
%\newcommand{\del}[1]{}

\newcommand{\note}[1]{{\sffamily\itshape\bfseries\uline{#1}}}

\newcommand{\CSLE}{Conflicting SRLG(Shared Risk Link Group) Link Exclusion Set }
\newcommand{\CSLI}{Conflicting SRLG(Shared Risk Link Group) Link Inclusion Set }
\newcommand{\CE}{$\mathbb{CSLE}$ }
\newcommand{\CI}{$\mathbb{CSLI}$ }

\newcommand{\CSLEs}{Conflicting SRLG(Shared Risk Link Group) Link Exclusion Set of graph $G^*$ }
\newcommand{\CSLIs}{Conflicting SRLG(Shared Risk Link Group) Link Inclusion Set of graph $G^*$ }
\newcommand{\CEs}{$\mathbb{CSLE}^*$ }
\newcommand{\CIs}{$\mathbb{CSLI}^*$ }

\newtheorem{theorem}{Theorem}
\newtheorem{lemma}{Lemma}
\newtheorem{definition}{Definition}



\begin{document}

%\begin{CJK*}{GBK}{song}
\title{One Node-Fault Tolerant Graph with Node's types for survivable Virtual Network Request with Service Type}
\author{Kun Xie$^1$, \emph{Member, IEEE},Heng Tao$^1$\\
$^1$ College of Computer Science and Electronic Engineering,Hunan University, China\\}

\maketitle
\vspace{-3em}
%\vspace{-0.6in}
\begin{abstract}
As virtualization is becoming a promising way to support various emerging application, provisioning survivability to requested
substrate networks (SN) which is embedded with virtual network in a resource efficient way is important. In this paper, First, we consider the
failure dependent protection (FDP) in which each primary facility node would have a different backup facility node, as opposed to the Failure Independent Protection (FIP) which has been studied before, in order to provide the same degree of protection against a single node failure with less substrate resources
\end{abstract}

\section{Introduction}
fff



\bibliographystyle{ieeetr}

\bibliography{BibNFT}

\end{document}
