\section{Related work}
Network virtualization is a promising technology to reduce the operating costs and management complexity of networks, and it is receiving an increasing amount of research interest[Network virtualization: State of the art and research challenges.]. Survivability is bound to become a more prominent issue as infrastructure providers move toward virtualizing their networks over cheaper commodity hardware [a platform for building flexible, fast virtual networks on commodity hardware.]. we are concerned with critical virtual nodes and embedding them as an entire infrastructure with Survivability guarantees.
Fault tolerance is provided in data centers [A scalable fault-tolerant layer 2
data center network fabric, A high performance, server-centric network architecture for modular data center] through excessive redundant nodes and links organized in a special way. These works provide Survivability but do not customize Survivability guarantees to embedded VInfs. However, such slice is not used as a back-up, but as a monitoring tool, and as a way to debug
the network in the case of failure. [Virtual routers on the move: live
router migration as a network-management primitive.] considered the use
of virtualized router as a management primitive that can be used to migrate routers for maximal reliability.

Meanwhile some works address node fault tolerance at the virtualization level. Bressoud [Hypervisor-based fault tolerance] was the first few to introduce fault tolerance at the hypervisor. Two virtual slices residing on the same physical node can be made to operate in sync through the hypervisor on the same physical node. Others [Live migration of virtual machines, Remus: High availability via asynchronous virtual machine replication.] have made progress for the virtual slices to be duplicated and migrated over a network. Various duplication techniques and migration protocols were proposed for different types of applications (web servers, game servers, and benchmarking applications) [Live migration of virtual machines]. Remus [Remus: High availability via asynchronous virtual machine replication.] and Kemari [Kemari: VM Synchronization for Fault Tolerance.] are two other systems that allows for state synchronization between two virtual nodes for full, dedicated
redundancy. However, these works focus on the practical issues, and do not address the resource allocation issue. VNsnap [VNsnap: Taking Snapshots of Virtual Networked Environments with Minimal Downtime.] is another method to take static snapshots of an entire virtual infrastructure to some reliable storage, in order to recover from failures. This can be stored as reliably
in a distributed manner as replicas [Bigtable: A Distributed Storage System for Structured Data.], or as parities [Decentralized erasure codes for
distributed networked storage.,Highly Available Virtual Machines with Network Coding.]. VNsnap does not address synchronization, nor guarantee sufficient resources for recovery from snapshots. Fundamentally, there are methods to construct topologies for redundant nodes that address both nodes and links reliability [Fault tolerant graphs,perfect hash functions and disjoint paths., Node-covering,error-correcting codes and multiprocessors with very high average fault tolerance]. Based on some input graph, additional links are introduced such that the least number is needed. However, a node failure, in this case, may involve migrations among the remaining nodes to preserve the original topology. This may not be suitable in a scenario where migrations
may disrupt other running parts of the network.

%Our problem involves virtual network embedding [8, 18]
%with added node and link redundancy for reliability. Our
%model employs the use of path-splitting [26], which allows a
%link to be split over multiple routes such that the aggregate
%flow across those routes equal to the demand between the
%two nodes. This gives more resilience to link failures and
%allows for graceful degradation. A related work that does
%not use path-splitting for embedding reliability is [16].
