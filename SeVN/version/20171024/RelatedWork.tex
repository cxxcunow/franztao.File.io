\section{Related work}



Meanwhile some works address node fault tolerance at the virtualization level. Bressoud \cite{bressoud1996hypervisor} was the first few to introduce fault tolerance at the hypervisor. Two virtual slices residing on the same physical node can be made to operate in sync through the hypervisor on the same physical node. Others \cite{wang2008virtual,cully2008remus} have made progress for the virtual slices to be duplicated and migrated over a network. Various duplication techniques and migration protocols were proposed for different types of applications (web servers, game servers, and benchmarking applications) \cite{wang2008virtual,cully2008remus} and Kemari \cite{tamura2008kemari} are two other systems that allows for state synchronization between two virtual nodes for full, dedicated
redundancy. However, these works focus on the practical issues, and do not address the resource allocation issue. VNsnap \cite{kangarlou2009vnsnap} is another method to take static snapshots of an entire virtual infrastructure to some reliable storage, in order to recover from failures. This can be stored as reliably
in a distributed manner as replicas \cite{chang2008bigtable}, or as parities \cite{dimakis2006decentralized,yeow2011highly}. VNsnap does not address synchronization, nor guarantee sufficient resources for recovery from snapshots. Fundamentally, there are methods to construct topologies for redundant nodes that address both nodes and links reliability \cite{ajtai1992fault,dutt1997node}. Based on some input graph, additional links are introduced such that the least number is needed. However, a node failure, in this case, may involve migrations among the remaining nodes to preserve the original topology. This may not be suitable in a scenario where migrations
may disrupt other running parts of the network.

%Our problem involves virtual network embedding [8, 18]
%with added node and link redundancy for reliability. Our
%model employs the use of path-splitting [26], which allows a
%link to be split over multiple routes such that the aggregate
%flow across those routes equal to the demand between the
%two nodes. This gives more resilience to link failures and
%allows for graceful degradation. A related work that does
%not use path-splitting for embedding reliability is [16].

%Zhang et al. [7,8] have considered substrate resource sharing among multiple virtual networks. However, their method only con- siders substrate resource sharing within the same priority class and does not consider sharing among different priority classes while at the same time satisfying the different latency require- ments. Thus, the method cannot be applied to the VNE problem where substrate resource sharing among multiple priority classes is required within each requested virtual network. This paper pro- poses a heuristic VNE method to minimize the required amount of substrate resources due to fair substrate resource sharing among multiple virtual networks, while considering the existence of mul- tiple priority classes that share the substrate resources with one another within each virtual network. Since substrate resource shar- ing among multiple virtual networks is expected to occur on the substrate link bandwidth more frequently, the proposed heuris- tic method prioritizes virtual link assignment rather than virtual node assignment, in contrast to most of the existing heuristic methods.


